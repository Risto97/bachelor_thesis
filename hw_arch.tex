\section{Arhitektura hardvera}

\subsection{Uvod}\label{hw_arch_intro}

U ovom poglavlju biće opisana projektovana arhitektura hardvera. \\
Predložena hardverska arhitektura će biti opisani uz što manje detalja
implementacije.
PyGears i SystemVerilog implementacije će pratiti predloženu arhitekturu, ali će
se razilikovati u detaljima. \\

Na slici(\ref{hw_arch_top}) prikazan je uprošćen blok dijagram realizovanog IP jezgra.
\begin{figure}[H]
\centering{
    \resizebox{\textwidth}{!}{%
    \pgfdeclarelayer{background}
\pgfdeclarelayer{foreground}
\pgfsetlayers{background,main,foreground}

\tikzstyle{cloud1} = [draw=black, thick, fill=red!20, minimum hegiht = 1em]
\tikzstyle{block_l} =[draw, text centered, fill=blue!15, minimum width=3cm, minimum height=3.5cm]
\tikzstyle{block_m} =[draw, text centered, fill=blue!15, minimum width=2.5cm, minimum height=2cm]
\tikzstyle{block_s} =[draw, text centered, fill=blue!15, minimum width=2cm, minimum height=1.5cm]
\tikzstyle{line} = [draw, decoration={markings,mark=at position 1 with {\arrow[scale=2.2,black]{>}}},
    postaction={decorate},
    shorten >=0.4pt]

\begin{tikzpicture}[thick]
  \node [block_l] (img_ram) {IMG RAM};
  \node [coordinate, above = 0cm of img_ram, yshift=-0.5cm, xshift=1.5cm] (img_ram_addr_in){};
  \node [coordinate, left =2.0cm of img_ram] (img_in){};
  \node [coordinate, left = 0cm of img_ram] (img_ram_img_in){};
  \node [coordinate, right = 0cm of img_ram] (img_ram_img_out){};

  \node [block_m, above right = +1cm and 0.2cm of img_ram] (rd_addrgen) {rd\_addrgen};
  \node [coordinate, below = 0cm of rd_addrgen] (rd_addr_addr_out){};

  \node [block_m, above right = -1.75cm and 2.5cm of img_ram ] (ii_gen) {ii\_gen};
  \node [coordinate, left = 0cm of ii_gen] (ii_gen_in){};
  \node [coordinate, right = 0cm of ii_gen] (ii_gen_out){};
  \node [block_m, below = 0.25cm of ii_gen ] (sii_gen) {sii\_gen};
  \node [coordinate, left = 0cm of sii_gen] (sii_gen_in){};
  \node [coordinate, right = 0cm of sii_gen] (sii_gen_out){};

  \node [block_s, right = 1.25cm of sii_gen ] (stddev) {stddev};
  \node [coordinate, above left = -0.25cm and 0cm of stddev] (stddev_ii){};
  \node [coordinate, above left = -0.75cm and 0cm of stddev] (stddev_sii){};
  \node [coordinate, right = 0cm of stddev] (stddev_out){};

  \node [block_m, right = 1.0cm of ii_gen ] (frame_buffer) {frame\_buffer};
  \node [coordinate, left = 0cm of frame_buffer] (frame_buffer_in){};
  \node [coordinate, above left = -0.5cm and 0cm of frame_buffer] (frame_buffer_rect_addr){};
  \node [coordinate, right = 0cm of frame_buffer] (frame_buffer_out){};

  \node [block_m, above = 1.0cm of frame_buffer ] (features_mem) {features\_mem};
  \node [coordinate, left = 0cm of features_mem] (rects_addr){};
  \node [coordinate, right = 0cm of features_mem] (rects_weights){};

  \node [block_l, fill=red!15, above right = -3.5cm and 1cm of frame_buffer] (classifier) {\huge Classifier};
  \node [coordinate, above left = -1cm and 0cm of classifier] (classifier_ii){};
  \node [coordinate, above left = -2.5cm and 0cm of classifier] (classifier_stddev){};
  \node [coordinate, above left = -0.5cm and 0cm of classifier] (classifier_weights){};
  \node [coordinate, above right = -1.0cm and 0cm of classifier] (detected_addr){};
  \node [coordinate, above right = -2.5cm and 0cm of classifier] (irq){};

  %% group %%
 \begin{pgfonlayer}{background}
  \node[inner sep=7pt, fill=yellow!25, rounded corners, inner ysep =30pt, draw, thick,fit=(img_ram) (rd_addrgen) (ii_gen) (features_mem) (sii_gen) (frame_buffer) (stddev) (classifier)] (ip_core) {};
  \node[above left] at (ip_core.south east) {\huge Cascade Classifier IP core};
\end{pgfonlayer}

  % arrows
  \path [line] (rd_addr_addr_out)  |- (img_ram_addr_in) node[transition, xshift=0.8cm, yshift=0.25cm] {rd\_addr};
  \path [line] (img_in) node[transition, yshift=0.25cm, xshift=0.65cm]{\large img\_in} -- (img_ram_img_in);
  \path [line] (img_ram_img_out) node[transition, yshift=0.2cm, xshift=0.9cm] {img\_out} -- +(1.75cm, 0cm) |- (ii_gen_in);
  \path [line] (img_ram_img_out) -- +(1.75cm, 0cm) |- (sii_gen_in);
  \path [line] (ii_gen_out) -- +(0.5cm, 0cm) node[transition, yshift=0.2cm, xshift=0cm] {ii} |- (stddev_ii);
  \path [line] (sii_gen_out) -- +(0.5cm, 0cm)  node[transition, yshift=0.2cm, xshift=0cm] {sii} |- (stddev_sii);
  \path [line] (ii_gen_out) |- (frame_buffer);
  \path [line] (frame_buffer_out) node[transition, yshift=+0.2cm, xshift=0.5cm] {fb\_ii} |- (classifier_ii);
  \path [line] (stddev_out) -- +(0.5cm, 0) node[transition, yshift=-0.2cm, xshift=0.1cm] {stddev} |- (classifier_stddev);
  \path [line] (rects_addr) node[transition, yshift=+0.2cm, xshift=-1.1cm] {rects\_addr}  -- +(-0.5cm, 0cm) |- (frame_buffer_rect_addr);
  \path [line] (rects_weights) node[transition, yshift=+0.2cm, xshift=+1.2cm] {rects\_weights}  -- +(+0.5cm, 0cm) |- (classifier_weights);
  \path [line] (detected_addr) -- +(3.0cm, 0cm) node[transition, yshift=0.25cm, xshift=-1.1cm]{\large detect\_addr} ;
  \path [line] (irq) -- +(3.0cm, 0cm) node[transition, yshift=0.25cm, xshift=-0.4cm]{\large irq} ;
\end{tikzpicture}

  }}
\caption{Arhitektura hardvera kaskadnog klasifikatora}
\label{hw_arch_top}
\end{figure}

U nastavku će biti opisana arhitektura.
Prvo će biti opisani spoljni interfejsi, zatim će detaljnije biti opisane
interne komponente. \\

\subsection{Interfejsi IP jezgra}
\emph{IP} jezgro se povezuje ostatkom sistema pomoću 3 spoljna interfejsa:

\begin{itemize}
  \item Ulazni interfejs \textbf{img\_in} je 8-bitni podatak i predstavlja
    vrednosti piksela slike u \emph{grayscale} formatu (nijanse sive).
  \item Izlazni interfejs \textbf{detect\_addr} ukoliko jezgro detektuje objekat
    na slici, na ovom interfejsu će se pojaviti X i Y koordinate
    detektovanog objekta.
  \item Izlazni interfejs ili signal \textbf{irq} signalizira završetak obrade slike i
    daje do znanja ostatku sistema da je jezgro spremno za obradu nove slike.
    Namenjen da se koristi kao prekidni signal za procesor.
\end{itemize}

\subsection{Modul IMG RAM}

U toku rada IP jezgra postoji potreba za ponovnim čitanjem istih piksela slike.
Čitanje iz eksterne memorije može biti sporo i energetski neefikasno, kako bi se
smanjila potreba za višestrukim čitanjem istog podatka iz spoljne memorije,
odlučeno je da se cela slika se čuva u manju memoriju unutar IP jezgra. \\

Modul \textbf{IMG RAM} je zadužen za skladištenje slike koja se obrađuje.
Sastoji se od \gls{ram} memorije i brojača za generisanje adrese za upis. \\
Korišćena RAM memorija treba da ima jedan port za upis i jedan za čitanje.\\

Prilikom upisivanja slike u memoriju, posle svakog primljenog podatka na ulaznom
interfejsu brojač adrese upisa se poveća za 1. \\

Skladištena slika je u 8-bitnom formatu i predstavlja grayscale vrednost piksela
originalne slike. \\
Veličina memorije je $width*height*8$ bit. Za dimenziju slike 240x320
veličina memorije je $614400$ bita. \\

IMG RAM se povezuje sa ostatkom sistema preko 3 interfejsa:
\begin{itemize}
  \item Ulazni interfejs \textbf{img\_in} ekvivalentan je sa istoimenim
    interfejsom na višem nivou hijerarhije.
  \item Ulazni interfejs \textbf{rd\_addr} prihvata linearizovanu adresu za
    čitanje piksela slike iz ove memorije.
    Ova adrese je generisana od strane \textbf{rd\_addrgen} modula.
  \item Izlazni interfejs \textbf{img\_out} šalje pročitane vrednosti piksela iz
    RAM memorije ostatku sistema.
\end{itemize}

\newpage

\subsection{Modul rd\_addrgen}

Modul \textbf{rd\_addrgen} je zadužen za generisanje adrese za čitanje iz
\textbf{img\_ram} modula. \\
Blok dijagram ovog modula prikazan je na slici (\ref{rd_addrgen_bd}). \\

\begin{figure}[H]
    

\pgfdeclarelayer{background}
\pgfdeclarelayer{foreground}
\pgfsetlayers{background,main,foreground}

\tikzstyle{cloud1} = [draw=black, thick, fill=red!20, minimum height = 1em]
\tikzstyle{block_l} =[draw, text centered, fill=red!15, minimum width=2.5cm, minimum height=2.5cm]
\tikzstyle{block_m} =[draw, text centered, fill=blue!15, minimum width=2.5cm, minimum height=1.5cm]
\tikzstyle{block_s} =[draw, text centered, fill=blue!15, minimum width=2cm, minimum height=1.5cm]
\tikzstyle{line} = [draw, decoration={markings,mark=at position 1 with {\arrow[scale=2.2,black]{>}}},
    postaction={decorate},
    shorten >=0.4pt]

\begin{tikzpicture}[thick]
  \node [block_m] (scale_counter) {scale\_counter};
  \node [coordinate, right = 0cm of scale_counter] (scale_counter_out){};

  \node [block_m, above right = 0cm and 1.25cm of scale_counter] (boundaries) {boundaries};
  \node [coordinate, left = 0cm of boundaries] (boundaries_in){};
  \node [coordinate, right = 0cm of boundaries] (boundaries_out){};

  \node [block_m, below right = 0cm and 1.25cm of scale_counter] (scale_ratio) {scale\_ratio};
  \node [coordinate, left = 0cm of scale_ratio] (scale_ratio_in){};
  \node [coordinate, right = 0cm of scale_ratio] (scale_ratio_out){};

  \node [block_l, right = 2cm of boundaries] (hopper) {\Large hopper};
  \node [coordinate, left = 0cm of hopper] (hopper_boundaries_in){};
  \node [coordinate, below = 0cm of hopper] (hopper_out){};

  \node [block_l, right = 2cm of scale_ratio] (sweeper) {\Large sweeper};
  \node [coordinate, above = 0cm of sweeper] (sweeper_hop_in){};
  \node [coordinate, left = 0cm of sweeper] (sweeper_scale_in){};
  \node [coordinate, right = 0cm of sweeper] (sweeper_out){};

  \node [block_m, right = 1.25cm of sweeper] (addr_trans) {addr\_trans};
  \node [coordinate, left = 0cm of addr_trans] (addr_trans_in){};
  \node [coordinate, right = 0cm of addr_trans] (addr_trans_out){};

  %% group %%
 \begin{pgfonlayer}{background}
  \node[inner sep=7pt, fill=yellow!25, rounded corners, inner ysep =30pt, draw, thick,fit=(scale_counter) (boundaries) (scale_ratio) (hopper) (sweeper) (addr_trans)] (ip_core) {};
  \node[above left] at (ip_core.south east) {\huge rd\_addrgen};
\end{pgfonlayer}

  % arrows
  \path [line] (scale_counter_out) -- +(0.5cm, 0) node[transition, yshift=0cm, xshift=1.1cm] {scale\_num} |- (boundaries_in);
  \path [line] (scale_counter_out) -- +(0.5cm, 0) |- (scale_ratio_in);

  \path [line] (boundaries_out) node[transition, yshift=0.2cm, xshift=1.0cm] {boundary} -- (hopper_boundaries_in);
  \path [line] (scale_ratio_out) node[transition, yshift=0.2cm, xshift=0.8cm] {scale} -- (sweeper_scale_in);

  \path [line] (hopper_out) node[transition, yshift=-0.25cm, xshift=0.4cm] {hop} -- (sweeper_hop_in);

  \path [line] (sweeper_out) node[transition, yshift=0.2cm, xshift=0.6cm] {sweep} -- (addr_trans_in);
  \path [line] (addr_trans_out)-- +(1.5cm, 0) node[transition, yshift=0.3cm, xshift=-0.5cm] {rd\_addr} ;

\end{tikzpicture}
\caption{Blok dijagram \textbf{rd\_addrgen} modula}
\label{rd_addrgen_bd}
\end{figure}

U sklopu ovog modula rešeno je i skaliranje slike, opisano u sekciji \ref{image_scaling}.

\subsubsection{Scale\_counter}\label{scale_counter_sec}
\textbf{Scale\_counter} je brojač koji predstavlja trenutni nivo piramide
skaliranja prikazane na slici(\ref{image_pyramid}).
Nakom završetka obrade slike na trenutnom nivou piramide ovaj brojač se uveća za
1. \\

\subsubsection{Boundaries}\label{boundaries_sec}
Ulazni interfejs \textbf{Boundaries} modula je povezan sa izlazom
\textbf{scale\_counter} modula, odnosno dobija trenutni stepen skaliranja slike kao ulaz.
Na osnovu stepena skaliranje ovaj modul na izlazu daje granične vrednosti
\textbf{X} i \textbf{Y} koordinata za \textbf{hopper} modul.\\
Granice osiguravaju ispravno generisanje adrese, odnosno obezbeđuju da
adresa za čitanje iz \textbf{IMG RAM} modula nikad ne iskoči iz opsega memorije. \\
Ove vrednosti su fiksne i moguće ih je izračunati pre sinteze hardvera, pa su
zato softverski izračunate i prosleđene kao parametri. \\
Ovaj modul se sastoji od dva multipleksera i liste unapred poznatih konstanti. \\

\subsubsection{Scale\_ratio}\label{scale_ratio_sec}

\textbf{Scale\_ratio} je sličan \textbf{boundaries} modulu i prihvata isti
ulazni podata. \\
Razlika u odnosu na \textbf{boundaries} modul je u vrednosti parametara.
Parametri u ovom modulu se prosleđuju \textbf{sweeper} modulu i omogućavaju
računanje skalirane adrese pomoću aritmetike sa fiksnom tačkom. \\

\subsubsection{Hopper i Sweeper}\label{hopper_sec}
\textbf{Hopper} se može zamisliti kao dvostruka ugnježdena for petlja gde
iteratori petlje predstavljaju koordinate \textbf{X} i \textbf{Y}. \\
Prvo se iterira po \textbf{X} kordinati pa zatim po \textbf{Y}.

\begin{lstlisting}[language=C++,caption={Primer \textbf{hopper}-a u \textbf{C}-u},captionpos=b, label=hopper_code]
  for(int y = 0; y < boundary_y[scale_num]; y++){
    for(int x = 0; x < boundary_x[scale_num]; x++){
      // Sweeper
    }
  }
\end{lstlisting}

Na primeru (\ref{hopper_code}) prikazana je implementacija \textbf{hopper}-a u
\textbf{C}-u.
Može se videti da gornje granice \textbf{hopper}-a predstavljaju konstante
\textbf{boundary\_x[scale\_num]} i \textbf{boundary\_y[scale\_num]}, kao što je
opisano u sekcijama (\ref{boundaries_sec},
\ref{scale_counter_sec}). \\

Iteratori petlje se prosleđuju na izlazni interfejs \textbf{hopper}-a i služe
kao početne tačke za brojače \textbf{sweeper}-a koji će biti opisan u nastavku. \\

Slično kao hopper i sweeper se može predstaviti kao dve ugnježdene for petlje. \\
Ponovo se prvo iterira po X koordinati pa onda po Y.

\begin{lstlisting}[language=C++,caption={Primer \textbf{hopper}-a i \textbf{sweeper}-a u \textbf{C}-u},captionpos=b, label=sweeper_code]
  int x, y;
  // hopper
  for(int hop_y = 0; hop_y < boundary_y[scale_num]; hop_y++){
    for(int hop_x = 0; x < boundary_x[scale_num]; hop_x++){
      // sweeper
      for(y = hop_y; y < hop_y+feature_height; y++){
        for(x = hop_x; x < hop_x+feature_width; x++){
          // scale address
          // translate to linear address
        }
      }
    }
  }
\end{lstlisting}

Kao što se vidi na primeru (\ref{sweeper_code}) \textbf{hopper} i \textbf{sweeper} se zajedno mogu
predstaviti kao četiri ugnježdene for petlje. \\
\textbf{Sweeper} će kao donje granicu petlji uzeti trenutne vrednosti
\textbf{hopper} koordinata, zatim će se iterirati \textbf{feature\_width} i \textbf{feature\_height} puta. \\


Rad \textbf{hopper}-a i \textbf{sweeper}-a je nalakše opisati slikama, što je i
učinjeno na slikama(\ref{hop_sweep1}, \ref{hop_sweep2}).
Na slikama (\ref{hop_sweep1}, \ref{hop_sweep2}) prelazak \textbf{sweeper}-a po slici je
prikazan horizontalnim i dijagonalnim strelicama.
Plavim kružićima predstavljeni su pikseli koje sweeper prebriše za jedan
položaj hopper-a.\\

Gornji levi kružić predstavlja koordinatu trenutnog položaja hopper-a.
Crevnom strelicom je obeleženo iteriranje hopper-a kroz sliku. \\

\begin{figure}[H]
  \centering
    \resizebox{\textwidth}{!}{%
    
\pgfdeclarelayer{background}
\pgfdeclarelayer{foreground}
\pgfsetlayers{background,main,foreground}

\tikzstyle{cloud1} = [draw=black, thick, fill=red!20, minimum height = 1em]
\tikzstyle{block_l} =[draw, text centered, fill=red!15, minimum width=2.5cm, minimum height=2.5cm]
\tikzstyle{block_m} =[draw, text centered, fill=blue!15, minimum width=2.5cm, minimum height=1.5cm]
\tikzstyle{block_s} =[draw, text centered, fill=blue!15, minimum width=2cm, minimum height=1.5cm]
\tikzstyle{line} = [draw, thick, decoration={markings,mark=at position 1 with {\arrow[scale=2.2,black]{>}}},
    postaction={decorate},
    shorten >=0.4pt]
\tikzstyle{line_s} = [draw, ->]
\tikzstyle{line_st} = [draw,thick, ->]

\begin{tikzpicture}
  % draw the grid and the numbers
  \draw (-1,+1) grid (7,9) foreach \i in {0,...,7}{
    (\i-.5,9.5) node{\i} (-1.5,8.5-\i) node{\i}};

  % for every line we give first and last index to fill
  \foreach[count=\j] \a/\b in {0/4,0/4,0/4,0/4,0/4}
    \foreach \i in {\a,...,\b}
    \fill[blue!35,shift={(-.5,+9.5)},yscale=-1] (\i,\j) circle(2.0mm);

    \foreach \y in {8.5,...,5.5}{
      \foreach \x in {-0.5}{
        \path [line] (\x, \y) -- (\x+4, \y);
        \path [line_s] (\x+4, \y) -- (-.5, \y-1);%
      }}
    \path [line] (-0.5, 4.5) -- (-0.5+4, 4.5);
    \path [line_st] (+2.0, +10.0) node[transition, yshift=0.3cm, xshift=-0.5cm] {sweeper} -- (+2.0, 8.5);
    \path [line_st] (0.0, +10.0) node[transition, yshift=0.3cm, xshift=-0.5cm] {hopper} -- (-0.5, 8.5);

\end{tikzpicture}
    \pgfdeclarelayer{background}
\pgfdeclarelayer{foreground}
\pgfsetlayers{background,main,foreground}

\tikzstyle{cloud1} = [draw=black, thick, fill=red!20, minimum height = 1em]
\tikzstyle{block_l} =[draw, text centered, fill=red!15, minimum width=2.5cm, minimum height=2.5cm]
\tikzstyle{block_m} =[draw, text centered, fill=blue!15, minimum width=2.5cm, minimum height=1.5cm]
\tikzstyle{block_s} =[draw, text centered, fill=blue!15, minimum width=2cm, minimum height=1.5cm]
\tikzstyle{line} = [draw, thick, decoration={markings,mark=at position 1 with {\arrow[scale=2.2,black]{>}}},
    postaction={decorate},
    shorten >=0.4pt]
\tikzstyle{line_s} = [draw, ->]
\tikzstyle{line_st} = [draw,thick, ->]

\begin{tikzpicture}
  % draw the grid and the numbers
  \draw (-1,+1) grid (7,9) foreach \i in {0,...,7}{
    (\i-.5,9.5) node{\i} (-1.5,8.5-\i) node{\i}};

  % for every line we give first and last index to fill
  \foreach[count=\j] \a/\b in {0/4,0/4,0/4,0/4,0/4}
    \foreach \i in {\a,...,\b}
    \fill[blue!35,shift={(+0.5,+9.5)},yscale=-1] (\i,\j) circle(2.0mm);

    \foreach \y in {8.5,...,5.5}{
      \foreach \x in {-0.5}{
        \path [line] (\x+1, \y) -- (\x+5, \y);
        \path [line_s] (\x+5, \y) -- (+.5, \y-1);%
      }}
    \path [line] (+0.5, 4.5) -- (+0.5+4, 4.5);

    \path [line_st] (+3.0, +10.0) node[transition, yshift=0.3cm, xshift=-0.5cm] {sweeper} -- (+3.0, 8.5);

    \path [line_st] (+1.0, +10.0) node[transition, yshift=0.3cm, xshift=-0.5cm] {hopper} -- (+0.5, 8.5);

    % \draw[red] (+0.5,8.5) arc (0:120:0.5);
    \draw[thick, red, -latex] ($(-0.5, +8.5) + (0:0cm and 1cm)$(P) arc
    (0:180:-0.5cm and 0.45cm);
\end{tikzpicture}
    }
\caption{Način rada \textbf{hopper} i \textbf{sweeper} komponenti.}
\label{hop_sweep1}
\end{figure}

Na slici (\ref{hop_sweep2}) može se videti trenutak kada hopper dostiže granicu
boundary\_x nakon čega prelazi u novi red, uvećava Y koordinatu a X koordinatu
postavlja na nulu. \\

Nakon što hopper dostigne obe granice boundary\_x i boundary\_y završen je
trenutni stepen skaliranja slike i potrebno je uvećati scale\_num za jedan.

\begin{figure}[H]
  \centering
  \resizebox{\textwidth}{!}{%
    \pgfdeclarelayer{background}
\pgfdeclarelayer{foreground}
\pgfsetlayers{background,main,foreground}

\tikzstyle{cloud1} = [draw=black, thick, fill=red!20, minimum height = 1em]
\tikzstyle{block_l} =[draw, text centered, fill=red!15, minimum width=2.5cm, minimum height=2.5cm]
\tikzstyle{block_m} =[draw, text centered, fill=blue!15, minimum width=2.5cm, minimum height=1.5cm]
\tikzstyle{block_s} =[draw, text centered, fill=blue!15, minimum width=2cm, minimum height=1.5cm]
\tikzstyle{line} = [draw, thick, decoration={markings,mark=at position 1 with {\arrow[scale=2.2,black]{>}}},
    postaction={decorate},
    shorten >=0.4pt]
\tikzstyle{line_s} = [draw, ->]
\tikzstyle{line_st} = [draw,thick, ->]

\begin{tikzpicture}
  % draw the grid and the numbers
  \draw (-1,+1) grid (7,9) foreach \i in {0,...,7}{
    (\i-.5,9.5) node{\i} (-1.5,8.5-\i) node{\i}};

  % for every line we give first and last index to fill
  \foreach[count=\j] \a/\b in {0/4,0/4,0/4,0/4,0/4}
    \foreach \i in {\a,...,\b}
    \fill[blue!35,shift={(+2.5,+9.5)},yscale=-1] (\i,\j) circle(2.0mm);

    \foreach \y in {8.5,...,5.5}{
      \foreach \x in {-0.5}{
        \path [line] (\x+3, \y) -- (\x+7, \y);
        \path [line_s] (\x+7, \y) -- (+.5+2, \y-1);%
      }}
    \path [line] (+2.5, 4.5) -- (+0.5+6, 4.5);

    \path [line_st] (+5.0, +10.0) node[transition, yshift=0.3cm, xshift=-0.5cm] {sweeper} -- (+4.5, 8.5);

    \path [line_st] (+2.0, +10.0) node[transition, yshift=0.3cm, xshift=-0.5cm] {hopper} -- (+2.5, 8.5);

    % \draw[red] (+0.5,8.5) arc (0:120:0.5);
    \draw[thick, red, -latex] ($(+1.5, +8.5) + (0:0cm and 1cm)$(P) arc
    (0:180:-0.5cm and 0.45cm);
\end{tikzpicture}
    
\pgfdeclarelayer{background}
\pgfdeclarelayer{foreground}
\pgfsetlayers{background,main,foreground}

\tikzstyle{cloud1} = [draw=black, thick, fill=red!20, minimum height = 1em]
\tikzstyle{block_l} =[draw, text centered, fill=red!15, minimum width=2.5cm, minimum height=2.5cm]
\tikzstyle{block_m} =[draw, text centered, fill=blue!15, minimum width=2.5cm, minimum height=1.5cm]
\tikzstyle{block_s} =[draw, text centered, fill=blue!15, minimum width=2cm, minimum height=1.5cm]
\tikzstyle{line} = [draw, thick, decoration={markings,mark=at position 1 with {\arrow[scale=2.2,black]{>}}},
    postaction={decorate},
    shorten >=0.4pt]
\tikzstyle{line_s} = [draw, ->]
\tikzstyle{line_st} = [draw,thick, ->]

\begin{tikzpicture}
  % draw the grid and the numbers
  \draw (-1,+1) grid (7,9) foreach \i in {0,...,7}{
    (\i-.5,9.5) node{\i} (-1.5,8.5-\i) node{\i}};

  % for every line we give first and last index to fill
  \foreach[count=\j] \a/\b in {0/4,0/4,0/4,0/4,0/4}
    \foreach \i in {\a,...,\b}
    \fill[blue!35,shift={(-0.5,+8.5)},yscale=-1] (\i,\j) circle(2.0mm);

    \foreach \y in {7.5,...,4.5}{
      \foreach \x in {-0.5}{
        \path [line] (\x+0, \y) -- (\x+4, \y);
        \path [line_s] (\x+4, \y) -- (-.5, \y-1);
        \begin{clean}

        \end{clean}
1);%
      }}
    \path [line] (-0.5, 3.5) -- (+3.5, 3.5);

    \path [line_st] (+3.0, +10.0) node[transition, yshift=0.3cm, xshift=-0.5cm] {sweeper} -- (+2.5, 7.5);

    \path [line_st] (0, +10.0) node[transition, yshift=0.3cm, xshift=-0.5cm] {hopper} -- (-0.5, 7.5);

    % \draw[red] (+0.5,8.5) arc (0:120:0.5);
    \draw [line_st, red, very thick] (2.5, 8.5) -- (-.5, 7.5)
\end{tikzpicture}

    }
\caption{Prelazak \textbf{hopper}-a u novi red.}
\label{hop_sweep2}
\end{figure}

\subsubsection{Skaliranje adrese}\label{address_scaling_sec}

Unutar sweeper-a je implementirano i skaliranje adrese u svrhu skaliranja slike
objašnjeno u sekciji \ref{image_scaling}. \\
Kako bi se adresa skalirala potrebno je množiti adresu sa decimalnim faktorom (npr. $1.2$, $1.33$).
kako je u hardveru množenje sa decimalnim brojevima sa pokretnom tačkom skupa operacija,
množenje sa fiksnom tačkom je efikasnije. \\
To je odrađeno tako što je celobrojni faktor za množenje unapred softverski izračunat i smešten
u scale\_ratio modul opisan ranije, isto tako je i broj pomeranja u desno dobijene binarne
vrednosti unapred poznat.
Na ovaj način je ušteđeno dosta hardverskih resursa. \\


\begin{figure}[H]
  \centering
  \resizebox{.4\textwidth}{!}{%
    \pgfdeclarelayer{background}
\pgfdeclarelayer{foreground}
\pgfsetlayers{background,main,foreground}

\tikzstyle{cloud1} = [draw=black, thick, fill=red!20, minimum height = 1em]
\tikzstyle{block_l} =[draw, text centered, fill=red!15, minimum width=2.5cm, minimum height=2.5cm]
\tikzstyle{block_m} =[draw, text centered, fill=blue!15, minimum width=2.5cm, minimum height=1.5cm]
\tikzstyle{block_s} =[draw, text centered, fill=blue!15, minimum width=2cm, minimum height=1.5cm]
\tikzstyle{line} = [draw, thick, decoration={markings,mark=at position 1 with {\arrow[scale=2.2,black]{>}}},
    postaction={decorate},
    shorten >=0.4pt]
\tikzstyle{line_s} = [draw, ->]
\tikzstyle{line_st} = [draw,thick, ->]

\begin{tikzpicture}
  % draw the grid and the numbers
  \draw (-1,-1) grid (9,9) foreach \i in {0,...,9}{
    (\i-.5,9.5) node{\i} (-1.5,8.5-\i) node{\i}};

  \foreach \x in {-1,0,1,3,4,5, 7,8}
  \foreach \y in {-1,0,2,3,4,6,7,8}
  {
    \fill[blue!50] (\x+0.5,\y+0.5) circle (2.0mm);
  }
}

\end{tikzpicture}
  }
  \caption{Posledica skaliranja adrese.}
  \label{hop_sweep_scale}
\end{figure}

Ovako skalirana adresa prikazana je na slici (\ref{hop_sweep_scale}).
Može se videti da će u ovom slučaju svaka četvrta tačka biti preskočena.
Na taj način će se dobiti manja slika od originalne, u ovom primeru od početne
slike $10*10$ dobija se slika $8*8$. \\
Na taj način objekti koji su izgledali veći na originalnoj slici će izgledati
manji na skaliranoj slici, što nam je potrebno kako bi dobili nezavisnost
detekcije od veličine objekta, opisano u sekciji \ref{image_scaling}.

\subsubsection{Modul addr\_trans}\label{addr_trans_sec}

Zamišljanje slike kao dvodimenzionalne matrice je intuitivno.
Međutim podaci u RAM memoriji su složeni u jednodimenzionalnom nizu i tako se
trebaju adresirati. \\
Kako bi se ova neusklađenost rešila potrebrebno je linearizovati adresu dobijenu
iz prethodnih modula. \\
Odnosno adresu u formatu (y, x) treba pretvoriti u linearnu pomoću sledeće formule:

\begin{equation}
  \Scale[1.2]{lin\_addr = (y * img\_width) + x}
  \label{IntegralImage_eq2}
\end{equation}

\noindent
Gde su y i x koordinate iz sweeper-a a img\_width je parametar koji označava
širinu slike.
Operaciju skaliranja obavlja modul \textbf{addr\_trans}. \\

\subsection{Modul ii\_gen i sii\_gen} \label{ii_sii_gen_sec}

\subsubsection{Odabir algoritma}\label{ii_alg_sel_sec}

Jedan od kritičnih delova Viola-Jones algoritma je generisanje integralne slike
opisane u poglavlju \ref{ii_sec}. \\
Ispostavlja da i u hardverskoj implementaciji generisanje
integralne slike ima veliki uticaj na performanse i potrebne hardverske resurse sistema. \\

Možemo izabrati sekvencijalni ili paralelni algoritam. \\
Ukoliko bi se odabrao paralelni algoritam koji može da računa više piksela u paraleli
povećanje resursa bi se drastično odrazilo na \textbf{img\_ram} i \textbf{frame\_buffer} memorije.
Ali bi dobili bolje performanse sistema. \\
Kako bi implementacija paralelnog algoritma povećala kompleksnost ne samo ovog
modula nego i okolnih komponenti, u ovom projektu paralelni algoritam neće biti razmatran.

\subsubsection{Sekvencijalna implementacija generatora integralne slike}\label{ii_seq_alg_sec}

Zbog jednostavnosti i potrebom za malim brojem resursa za ovu arhitekturu
odabran je sekvencijalni algoritam generisanja integralne slike koji odgovara
jednačini(\ref{IntegralImage_eq2}) iz sekcije \ref{ii_sec}.\\

Prednosti ovog algoritma u odnosu na paralelni su manji memorijski zahtevi na
ulaznom i izlaznom interfejsu, potrebno je manje funkcionalnih jedinica i
unutrašnje memorije.
Dok je  mana je manja brzina računanja. \\
Ovaj algoritam izračunava po jedan piksel svaki takt. \\

\begin{figure}[H]
  \centering
  \resizebox{\textwidth}{!}{%
    \pgfdeclarelayer{background1}
\pgfdeclarelayer{background2}
\pgfdeclarelayer{foreground}
\pgfsetlayers{background1, background2,main,foreground}

\tikzstyle{cloud1} = [draw=black, thick, fill=red!20, minimum height = 1em]
\tikzstyle{block_l} =[draw, text centered, fill=red!15, minimum width=2.5cm, minimum height=2.5cm]
\tikzstyle{block_m} =[draw, text centered, fill=blue!15, minimum width=2.5cm, minimum height=1.5cm]
\tikzstyle{block_s} =[draw, text centered, fill=blue!15, minimum width=2cm, minimum height=1.5cm]
\tikzstyle{line} = [draw, arrows={-Triangle[length=0.2cm]}]

\tikzstyle{sum} = [draw, thick, fill=blue!15, circle, node distance = 2cm]
\tikzstyle{dreg} = [draw, text centered, fill=blue!15, minimum width = 1.5cm,
minimum height=1.5cm]
\tikzstyle{fifo} = [draw, text centered, fill=blue!15, minimum width = 2.5cm,
minimum height=1.5cm]

\begin{tikzpicture}[thick]
  % \node [block_m] (scale_counter) {scale\_counter};
  \node [sum] (accum_sum) {+};
  \node [coordinate, left = 0cm of accum_sum] (accum_sum_in1){};
  \node [coordinate, left = 3cm of accum_sum] (ii_gen_in){};
  \node [coordinate, above = 0cm of accum_sum] (accum_sum_in2){};
  \node [coordinate, right = 0cm of accum_sum] (accum_sum_out){};

  \node [dreg, right = 1cm of accum_sum] (accum_dreg) {dreg};
  \node [coordinate, left = 0cm of accum_dreg] (accum_dreg_in){};
  \node [coordinate, right = 0cm of accum_dreg] (accum_dreg_out){};
  \node [coordinate, above right = 1cm and 1cm of accum_dreg] (feedback_point_1){};

  \node [sum, right = 2cm of accum_dreg] (add2) {+};
  \node [coordinate, left = 0cm of add2] (add2_in1){};
  \node [coordinate, below = 0cm of add2] (add2_in2){};
  \node [coordinate, right = 0cm of add2] (add2_out){};
  \node [coordinate, right = 3.5cm of add2] (exit_point){};

  \node [fifo, below right = 1cm and 0.5cm of add2] (fifo) {FIFO};
  \node [coordinate, left = 0cm of fifo] (fifo_out){};
  \node [coordinate, right = 0cm of fifo] (fifo_in){};


  %% group %%
 \begin{pgfonlayer}{background2}
   \node[inner sep=10pt, fill=red!15, rounded corners, draw, thick,fit=(accum_sum) (accum_dreg) (feedback_point_1)] (accum) {};
  \node[above left] at (accum.south east) {\textbf{accum}};
\end{pgfonlayer}

 \begin{pgfonlayer}{background1}
   \node[inner sep=20pt, fill=yellow!25, rounded corners, draw, thick,fit=(accum) (add2) (fifo) (exit_point)] (ii_gen) {};
  \node[above left, yshift=-1.1cm] at (ii_gen.north east) {\huge \textbf{ii\_gen}};
\end{pgfonlayer}


  % arrows
  \path [line] (accum_sum_out) -- (accum_dreg_in);
  \path [line] (ii_gen_in) node[transition, yshift=0.2cm, xshift=0.6cm] {\textbf{img\_in}} -- (accum_sum_in1);
  \path [line] (accum_dreg_out) -- (add2_in1);
  \path [line] (accum_dreg_out) -|  (feedback_point_1) -| (accum_sum_in2);
  \path [line] (add2_out) -- (exit_point) |- (fifo_in);

  \path [line] (fifo_out) -| (add2_in2);
  \path [line] (add2_out) -- +(6cm,0cm) node[transition, yshift=0.2cm, xshift=-0.6cm] {\textbf{ii\_out}};
\end{tikzpicture}

    }
\caption{Blok dijagram ii\_gen modula}
\label{ii_gen}
\end{figure}

Na slici(\ref{ii_gen}) prikazana je uprošćena šema generatora integralne slike.
\\

Ulazni interfejs ovom modula se povezuje izlaznim interfejsom IMG RAM memorije.
\\

Pikseli u istom redu se akumuliraju pomoću sabirača i registra unutar accum modula.
Na slici je izostavljeno da se registar dreg resetuje posle dolaska poslednjeg
piksela u redu prozora.
Ovo je moguće dodavanjem dodatnih polja uz vrednost ulaznog piksela, koji
predstavljaju kraj transakcije za red i kolonu.
O ovom konceptu će biti više reči u sekciji posvećenoj PyGears tipovima. \\

Nakon toga akumulirana vrednost se sabira sa sadržajem FIFO bafera koji sadrži
vrednost piksela integralne slike iz predhodnog reda.
Zatim se vrednost prosleđuje na izlaz i ponovo upisuje u FIFO bafer.
Iz FIFO bafer-a će na ovaj način izlaziti vrednosti piksela iz prethodnog reda,
a bafer će se puniti vrednostima trenutnog reda. \\

\noindent
U ovom slučaju standardan FIFO bafer je modifikovan na sledeći način:
\begin{itemize}
  \item Dodat je PRELOAD parametar koji pomera pokazivač za upis na vrednost
    širine prozora (feature\_width) prilikom reseta. Ovo je potrebno da bi se
    obezbedilo čitanje nula iz bafera kada se obrađuje prvi red slike.
  \item FIFO se resetuje kada je završeno računanje celog prozora.
\end{itemize}

\subsubsection{Generator kvadratne integralne slike}\label{sii_gen_sec}

Generator kvadratne integralne slike je potreban za računanje standardne
devijacije prozora što je opisano u sekciji \ref{lumi_inv_sec}. \\
Generisanje kvadratne integralne slike je jednostavno uz korišćenje generatora
integralne slike.
Potrebno je kvadrirati ulazne piksele i dovesti ih na generator integralne slike
kao na slici(\ref{sii_gen}).

\begin{figure}[H]
  \centering
  \resizebox{0.7\textwidth}{!}{%
    \pgfdeclarelayer{background1}
\pgfdeclarelayer{background2}
\pgfdeclarelayer{foreground}
\pgfsetlayers{background1, background2,main,foreground}

\tikzstyle{cloud1} = [draw=black, thick, fill=red!20, minimum height = 1em]
\tikzstyle{block_l} =[draw, text centered, fill=red!15, minimum width=3.5cm, minimum height=3.5cm]
\tikzstyle{block_m} =[draw, text centered, fill=blue!15, minimum width=2.5cm, minimum height=1.5cm]
\tikzstyle{block_s} =[draw, text centered, fill=blue!15, minimum width=2cm, minimum height=1.5cm]
\tikzstyle{line} = [draw, arrows={-Triangle[length=0.2cm]}]

\tikzstyle{sum} = [draw, thick, fill=blue!15, circle, node distance = 2cm]
\tikzstyle{square} = [draw, thick, fill=blue!15, circle, node distance = 2cm]
\tikzstyle{dreg} = [draw, text centered, fill=blue!15, minimum width = 1.5cm,
minimum height=1.5cm]
\tikzstyle{fifo} = [draw, text centered, fill=blue!15, minimum width = 2.5cm,
minimum height=1.5cm]

\begin{tikzpicture}[thick]
  \node [square] (square) {\textbf{ $X^2$}};
  \node [coordinate, left = 0cm of square] (square_in);
  \node [coordinate, right = 0cm of square] (square_out);

  \node [block_l, rounded corners, right = 1cm of square] (ii_gen) {\Huge ii\_gen};
  \node [coordinate, left = 0cm of ii_gen] (ii_gen_in);
  \node [coordinate, right = 0cm of ii_gen] (ii_gen_out);

%   %% group %%
 \begin{pgfonlayer}{background2}
   \node[inner sep=25pt, fill=yellow!25, rounded corners, draw, thick,fit=(square) (ii_gen) ] (accum) {};
  \node[above left, xshift=3.2cm, yshift=-1.1cm] at (accum.north west) {\Huge \textbf{sii\_gen}};
\end{pgfonlayer}


  % arrows
  \path [line] (-3cm, 0cm) node[transition, yshift=0.2cm, xshift=0.6cm] {\textbf{img\_in}}  -- (square_in);
  \path [line] (square_out) -- (ii_gen_in);
  \path [line] (ii_gen_out) -- +(3cm, 0cm) node[transition, yshift=0.3cm, xshift=-0.6cm] {\textbf{sii\_out}};
\end{tikzpicture}
    }
\caption{Blok dijagram ii\_gen modula}
\label{sii_gen}
\end{figure}

\subsection{Modul frame\_buffer}

Potrebno je obezbediti da se generisana integralna slika može pročitati od
strane klasifikatora u nasumičnom maniru i da vrednost bude dostupna u roku od
jednog takta.
Kako bi se to obezbedilo potrebno je skladištiti integralnu sliku u lokalnu RAM
memoriju. \\

\begin{figure}[H]
  \centering
  \resizebox{0.85\textwidth}{!}{%
    \pgfdeclarelayer{background}
\pgfdeclarelayer{foreground}
\pgfsetlayers{background,main,foreground}

\tikzstyle{cloud1} = [draw=black, thick, fill=red!20, minimum height = 1em]
\tikzstyle{block_l} =[draw, text centered, fill=blue!15, minimum width=2.5cm, minimum height=2.5cm]
\tikzstyle{block_m} =[draw, text centered, fill=blue!15, minimum width=2.5cm, minimum height=1.5cm]
\tikzstyle{block_s} =[draw, text centered, fill=blue!15, minimum width=2cm, minimum height=1.5cm]
\tikzstyle{line} = [draw, arrows={-Triangle[length=0.2cm]}]

\begin{tikzpicture}[thick]
  \node [block_l] (ram0) {RAM0};
  \node [coordinate, above left = -0.75cm and 0cm of ram0] (ram0_rect_addr){};
  \node [coordinate, right = 0cm of ram0] (ram0_out){};
  \node [coordinate, right = 3cm of ram0_out] (ram0_out_point){};
  \node [coordinate, left = 3cm of ram0_rect_addr] (ram0_rect_addr_point){};
  \node [coordinate, below left = -0.75cm and 0cm of ram0] (ram0_ii){};
  \node [coordinate, left = 2cm of ram0_ii] (ram0_ii_point){};

  \node [block_l, below = 1cm of ram0] (ram1) {RAM1};
  \node [coordinate, right = 0cm of ram1] (ram1_out){};
  \node [coordinate, right = 3cm of ram1_out] (ram1_out_point){};
  \node [coordinate, right = 3cm of ram1_out_point] (rect_data_out){};
  \node [coordinate, above left = -0.75cm and 0cm of ram1] (ram1_rect_addr){};
  \node [coordinate, left = 3cm of ram1_rect_addr] (ram1_rect_addr_point){};
  \node [coordinate, left = 5.5cm of ram1_rect_addr] (rect_addr_group){};
  \node [coordinate, below left = -0.75cm and 0cm of ram1] (ram1_ii){};
  \node [coordinate, left = 2cm of ram1_ii] (ram1_ii_point){};


  \node [block_l, below = 1cm of ram1] (ram2) {RAM2};
  \node [coordinate, right = 0cm of ram2] (ram2_out){};
  \node [coordinate, right = 3cm of ram2_out] (ram2_out_point){};
  \node [coordinate, above left = -0.75cm and 0cm of ram2] (ram2_rect_addr){};
  \node [coordinate, left = 3cm of ram2_rect_addr] (ram2_rect_addr_point){};
  \node [coordinate, below left = -0.75cm and 0cm of ram2] (ram2_ii){};
  \node [coordinate, left = 2cm of ram2_ii] (ram2_ii_point){};

  % \node [block_m, above right = 0cm and 1.25cm of scale_counter] (boundaries) {boundaries};
  % \node [coordinate, left = 0cm of boundaries] (boundaries_in){};
  % \node [coordinate, right = 0cm of boundaries] (boundaries_out){};

  % \node [block_m, below right = 0cm and 1.25cm of scale_counter] (scale_ratio) {scale\_ratio};
  % \node [coordinate, left = 0cm of scale_ratio] (scale_ratio_in){};
  % \node [coordinate, right = 0cm of scale_ratio] (scale_ratio_out){};

  % \node [block_l, right = 2cm of boundaries] (hopper) {\Large hopper};
  % \node [coordinate, left = 0cm of hopper] (hopper_boundaries_in){};
  % \node [coordinate, below = 0cm of hopper] (hopper_out){};

  % \node [block_l, right = 2cm of scale_ratio] (sweeper) {\Large sweeper};
  % \node [coordinate, above = 0cm of sweeper] (sweeper_hop_in){};
  % \node [coordinate, left = 0cm of sweeper] (sweeper_scale_in){};
  % \node [coordinate, right = 0cm of sweeper] (sweeper_out){};

  % \node [block_m, right = 1.25cm of sweeper] (addr_trans) {addr\_trans};
  % \node [coordinate, left = 0cm of addr_trans] (addr_trans_in){};
  % \node [coordinate, right = 0cm of addr_trans] (addr_trans_out){};

  %% group %%
 \begin{pgfonlayer}{background}
  \node[inner sep=15pt, fill=yellow!25, rounded corners, inner ysep =30pt, draw, thick,fit=(ram0) (ram1) (ram2)(ram0_rect_addr_point)(ram1_rect_addr_point)(ram2_rect_addr_point)(ram2_out_point) (ram1_out_point)(ram0_out_point) ] (frame_buffer) {};
  \node[above left] at (frame_buffer.south east) {\huge \textbf{frame\_buffer}};
\end{pgfonlayer}

  \node [coordinate, left = 0cm of frame_buffer, yshift=4cm] (ii_in_group){};
  \node [coordinate, left = 1.95cm of ii_in_group] (ii_in_group_point){};

  % arrows
  \path [line] (ram1_rect_addr_point) -| (ram0_rect_addr_point) node[transition, yshift=0.2cm, xshift=1.6cm] {rect\_addr[0]} -- (ram0_rect_addr);
  \path [line] (ram1_rect_addr_point) node[transition, yshift=0.2cm, xshift=1.6cm] {rect\_addr[1]}  -- (ram1_rect_addr);
  \path [line] (ram1_rect_addr_point) -| (ram2_rect_addr_point) node[transition, yshift=0.2cm, xshift=1.6cm] {rect\_addr[2]}  -- (ram2_rect_addr);
  \path [line, very thick] (rect_addr_group) node[transition, yshift=0.2cm, xshift=0.9cm] {\textbf{rect\_addr}}  -- (ram1_rect_addr_point);


  \path [line] (ram0_ii_point) node[transition, yshift=+0.3cm, xshift=0.8cm] {ii\_in}  -- (ram0_ii);
  \path [line] (ram1_ii_point) node[transition, yshift=+0.3cm, xshift=0.8cm] {ii\_in}   -- (ram1_ii);
  \path [line] (ram2_ii_point) node[transition, yshift=+0.3cm, xshift=0.8cm] {ii\_in}   -- (ram2_ii);

  \path [line] (ram0_out) -- (ram0_out_point) node[transition, yshift=0.3cm, xshift=-1.5cm] {rect\_data[0]}  -- (ram1_out_point);
  \path [line] (ram1_out) -- (ram1_out_point) node[transition, yshift=0.3cm, xshift=-1.5cm] {rect\_data[1]} ;
  \path [line] (ram2_out) -- (ram2_out_point) node[transition, yshift=0.3cm, xshift=-1.5cm] {rect\_data[2]} -- (ram1_out_point);
  \path [line, very thick] (ram1_out_point) -- (rect_data_out) node[transition, yshift=0.3cm, xshift=-1.2cm] {\textbf{rect\_data}} ;

  \path [line, very thick] (ii_in_group_point) node[transition, yshift=0.3cm, xshift=+0.5cm] {\textbf{ii\_in}}  -- (ii_in_group);
%   \path [line] (sweeper_out) node[transition, yshift=0.2cm, xshift=0.6cm] {sweep} -- (addr_trans_in);
%   \path [line] (addr_trans_out)-- +(1.5cm, 0) node[transition, yshift=0.3cm, xshift=-0.5cm] {rd\_addr} ;

\end{tikzpicture}
    }
\caption{Blok dijagram frame\_buffer-a realizovanog sa 3 jedno-portne RAM memorije.}
\label{frame_buffer_bd}
\end{figure}

U ovu svrhu je projektovana komponenta frame\_buffer.
Sastoji se od brojača za adresu upisa i RAM memorija.
Brojač adrese upisa se inkremetuje za jedan nakon svakog primljenog podatka kao
i kod IMG RAM modula.
Brojač nije prikazan na blok dijagramu iznad. \\
Potrebna veličina RAM memorije je data sledećim vezama:

\begin{figure}[H]
  \begin{subfigure}
    \begin{equation}
      size[bit] = frame\_width * frame\_height * w\_ii
      \label{frame_buffer_eq1}
    \end{equation}
  \end{subfigure}
  \hfill
  \begin{subfigure}
    \begin{equation}
      w\_ii = ceil(log_2(frame\_width * frame\_height * 2^{w\_img}))
      \label{frame_buffer_eq2}
    \end{equation}
  \end{subfigure}
\end{figure}

Gde su frame\_width i frame\_height širina i visina obeležja modela, w\_ii je
širina magistrale integralne slike, w\_img je ulazna širina magistrale piksela
slike. \\

Radi ubrzavanja rada klasifikatora možemo računati sva tri pravougaonika (\ref{haar_features_sec}) u paraleli.
Da bi se to obezbedilo potrebno je čitati u istom taktu tri vrednosti iz
frame\_buffer memorije.
Kako je više-portna memorija skupa i retko se nalazi u FPGA čipovima moguće
rešenje je koristiti tri dvo-portne memorije.
Ovo će kao rezultat zauzeti tri puta više RAM memorije na čipu, od koje će svaka
imati isti sadržaj.\\

Alat za sintezu uglavnom može da odradi transformaciju i da od jedne šesto-portne
memorije instancionira tri dvo-portne.
Ali se preporučuje da se ekslicitno instancioniraju tri memorije i pridržava
Synthesis Guideline-a npr. Xilinx \cite{XST}. \\
Primer prikazan na slici(\ref{frame_buffer_bd}).

\subsection{Modul stddev}

Računanje standardne devijacije prozora je potrebno kako bi se smanjio uticaj
različitog osvetljenja objekata na slikama. \\
Za ovo je zadužen modul stddev. U sledećem primeru prikazano je računanje
standardne devijacije u C++-u.

\begin{lstlisting}[language=C++,caption={Primer računanja standardne devijacije u \textbf{C}-u},captionpos=b, label=stddev_code]
  long calcStddev(long sii[FRAME_HEIGHT][FRAME_WIDTH],
                  long ii[FRAME_HEIGHT][FRAME_WIDTH]){

    long mean, stddev;

    mean = ii[0][0] + ii[FRAME_HEIGHT-1][FRAME_WIDTH-1] - ii[0][FRAME_WIDTH-1] - ii[FRAME_HEIGHT-1][0];

    stddev = sii[0][0] + sii[FRAME_HEIGHT-1][FRAME_WIDTH-1] - sii[0][FRAME_WIDTH-1] - sii[FRAME_HEIGHT-1][0];

    stddev = (stddev * (FRAME_WIDTH-1)*(FRAME_HEIGHT-1));
    stddev = stddev - (mean*mean);
    stddev = getSqrt(stddev);
    return stddev;
  }
\end{lstlisting}

Za računanje standardne devijacije potrebne su pikseli ivice prozora integralne i
kvadratne integralne slike, što se vidi u liniji 6 i 8
primera(\ref{stddev_code}). \\
Sabiranjem gornjeg levog i donjeg desnog piksela zatim oduzimanje gornjeg desnog
i donjeg levog integralne slike dobijamo sumu piksela celog prozora. \\
Ako pogledamo primer u C-u vidimo da imamo operacije sabiranja, oduzimanja,
kvadriranja promenljivih, zatim množenje sa konstantom. \\
Sa ovim operacijama većina alata za sintezu nema problem prilikom mapiranja i
većina FPGA čipova ima potrebne funkcionalne jedinice. \\

Konačno potrebno je odraditi kvadratni koren u liniji 12. \\
Ovo je operacija koju alati za automatsku sintezu ne mogu implementirati
na FPGA čipovima, pa je potrebno pronaći dobru aproksimaciju. \\

U ovom projektu je odlučeno koristiti lookup tabelu. \\
Za unapred definisani opseg vrednosti operanda za korenovanje softverski su
izračunate vrednosti kvadratnog korena i smeštene u listu. \\
Prilikom softverske analize odlučeno je da je 256 vrednosti korena dovoljno za ispravan
rad celog sistema, uz minimalan gubitak pouzdanosti. \\

\begin{figure}[H]
  \centering
  \resizebox{1\textwidth}{!}{%
    
\pgfdeclarelayer{background1}
\pgfdeclarelayer{background2}
\pgfdeclarelayer{foreground}
\pgfsetlayers{background1, background2,main,foreground}

\tikzstyle{cloud1} = [draw=black, thick, fill=red!20, minimum height = 1em]
\tikzstyle{block_l} =[draw, text centered, fill=blue!15, minimum width=2.5cm, minimum height=2.5cm]
\tikzstyle{block_m} =[draw, text centered, fill=blue!15, minimum width=2.0cm, minimum height=1.0cm]
\tikzstyle{block_s} =[draw, text centered, fill=blue!15, minimum width=2cm, minimum height=1.5cm]
\tikzstyle{line} = [draw, arrows={-Triangle[length=0.2cm]}]

\tikzstyle{dreg} = [draw, text centered, fill=blue!15, minimum width = 1.5cm,
minimum height=1.5cm]
\tikzstyle{frame_sum} = [draw, text centered, fill=blue!15, minimum width = 2.5cm,
minimum height=1.5cm]

\tikzstyle{square} = [draw, thick, fill=blue!15, circle, minimum size = 1cm, node distance = 2cm]
\tikzstyle{mult} = [draw, thick, fill=blue!15, circle, minimum size = 1cm ,node distance = 2cm]
\tikzstyle{sum} = [draw, thick, fill=blue!15, circle, minimum size= 1cm, node distance = 2cm]

\begin{tikzpicture}[thick]

  \node [frame_sum] (frame_sum_ii) {frame\_sum};
  \node [coordinate, left = 0cm of frame_sum_ii] (frame_sum_ii_in){};
  \node [coordinate, left = 2cm of frame_sum_ii_in] (ii_in){};
  \node [coordinate, right = 0cm of frame_sum_ii] (frame_sum_ii_out){};

  \node [square, right = 1.0cm of frame_sum_ii] (square_ii) {$X^2$};
  \node [coordinate, left = 0cm of square_ii] (ii_square_in){};
  \node [coordinate, right = 0cm of square_ii] (ii_square_out){};

  \node [frame_sum, below = 2cm of frame_sum_ii] (frame_sum_sii) {frame\_sum};
  \node [coordinate, left = 0cm of frame_sum_sii] (frame_sum_sii_in){};
  \node [coordinate, left = 2cm of frame_sum_sii_in] (sii_in){};
  \node [coordinate, right = 0cm of frame_sum_sii] (frame_sum_sii_out){};

  \node [mult, right = 1.0cm of frame_sum_sii] (mult_sii) {$*$};
  \node [coordinate, left = 0cm of mult_sii] (mult_sii_in1){};
  \node [coordinate, below = 0cm of mult_sii] (mult_sii_in2){};
  \node [coordinate, below = 1cm of mult_sii_in2] (mult_sii_const){};
  \node [coordinate, right = 0cm of mult_sii] (mult_sii_out){};

  \node [sum, below right = 0.75cm and 0.75cm of square_ii] (sub) {-};
  \node [coordinate, above = 0cm of sub] (sub_in1){};
  \node [coordinate, below = 0cm of sub] (sub_in2){};
  \node [coordinate, right = 0cm of sub] (sub_out){};

  \node [block_m, right = 0.75cm of sub] (shift) {$>>$};
  \node [coordinate, left = 0cm of shift] (shift_in){};
  \node [coordinate, right = 0cm of shift] (shift_out){};

  \node [block_l, right = 1.5cm of shift] (sqrt_rom) {SQRT\_ROM};
  \node [coordinate, left = 0cm of sqrt_rom] (addr_in){};
  \node [coordinate, right = 0cm of sqrt_rom] (sqrt_out){};
  \node [coordinate, right = 2cm of sqrt_out] (stddev_out){};

  %% group %%
 \begin{pgfonlayer}{background1}
   \node[inner sep=20pt, fill=yellow!25, rounded corners, draw, thick,fit=(frame_sum_ii) (frame_sum_sii) (mult_sii_const) (sqrt_rom)] (accum) {};
  \node[above left] at (accum.south east) {\huge \textbf{stddev}};
\end{pgfonlayer}


  % arrows
  \path [line] (frame_sum_ii_out) -- (ii_square_in);

  \path [line] (frame_sum_sii_out) -- (mult_sii_in1);
  \path [line] (mult_sii_const) -- (mult_sii_in2) node[transition, yshift=-1.3cm, xshift=+0.05cm] {\small $frame\_width*frame\_height$};

  \path [line] (ii_square_out) -| (sub_in1);
  \path [line] (mult_sii_out) -| (sub_in2);

  \path [line] (sub_out) -- (shift_in);
  \path [line] (shift_out) node[transition, yshift=0.7cm, xshift=+0.6cm] {\small sqrt\_addr} -- (addr_in);

  \path [line] (sqrt_out) -- (stddev_out) node[transition, yshift=0.3cm, xshift=-0.5cm] {\textbf{stddev}} ;

  \path [line] (ii_in) node[transition, yshift=0.2cm, xshift=+0.6cm] {\textbf{ii\_in}} -- (frame_sum_ii_in);
  \path [line] (sii_in) node[transition, yshift=0.2cm, xshift=+0.6cm] {\textbf{sii\_in}} -- (frame_sum_sii_in);

  % \path [line] (accum_sum_out) -- (accum_dreg_in);
  % \path [line] (ii_gen_in) node[transition, yshift=0.2cm, xshift=0.6cm] {\textbf{img\_in}} -- (accum_sum_in1);
  % \path [line] (accum_dreg_out) -- (add2_in1);
  % \path [line] (accum_dreg_out) -|  (feedback_point_1) -| (accum_sum_in2);
  % \path [line] (add2_out) -- (exit_point) |- (fifo_in);

  % \path [line] (fifo_out) -| (add2_in2);
  % \path [line] (add2_out) -- +(6cm,0cm) node[transition, yshift=0.2cm, xshift=-0.6cm] {\textbf{ii\_out}};
\end{tikzpicture}
    }
\caption{Blok dijagram stddev modula.}
\label{stddev_bd}
\end{figure}

Na slici(\ref{stddev_bd}) prikazan je blok dijagram projektovanog hardverskog
modula na osnovu primera(\ref{stddev_code}}).
Operaciju sabiranja piksela unutar prozora obavljaju frame\_sum moduli.
Mogu se videti i operator kvadriranja, množenja zatim i oduzimanja kao u
prethodnom primeru. \\
Lookup tabela je implementirana kao \gls{rom} memorija, pod nazivom SQRT\_ROM i
sadrži 256 memorijskih lokacija.

\subsection{Modul features\_mem} \label{features_meme_sec}

Svako obeležje u modelu se sastoji od najviše tri pravougaonika što je
objašnjeno u sekciji \ref{haar_features_sec}. \\
Svaki pravougaonik je definisan sa četiri koordinate (A, B, C, D) i sa težinom
njegove površine. \\
Zbog čestog i ponovljenog pristupa ovim obeležjima potrebno ih je skladištiti u
lokalnoj memoriji. \\

Zbog uštede memorije u hardverskoj implementaciji, moguće je svaki pravougaonik predstaviti koordinatom jedne njegove tačke zatim
širinom i visinom pravougaonika.
Ostale tačke je moguće izračunati pomoću ovih podataka. \\
Na ovaj način se vrši ušteda memorije, ali se uvodi potreba za množačima i
sabiračima za računanje ostalih koordinata.
U ovom slučaju zbog velikog broja obeležja (2913 u testiranom modelu \ref{haarcascade_frontal_sec}) ušteda
memorije je značajna, a dodatni sabirači i množači ne unose veliku cenu u sistem.

\begin{figure}[H]
  \centering
  \resizebox{0.6\textwidth}{!}{%
    
\pgfdeclarelayer{background}
\pgfdeclarelayer{foreground}
\pgfsetlayers{background,main,foreground}

\tikzstyle{cloud1} = [draw=black, thick, fill=red!20, minimum height = 1em]
\tikzstyle{block_l} =[draw, text centered, fill=red!15, minimum width=2.5cm, minimum height=2.5cm]
\tikzstyle{block_m} =[draw, text centered, fill=blue!15, minimum width=2.5cm, minimum height=1.5cm]
\tikzstyle{block_s} =[draw, text centered, fill=blue!15, minimum width=2cm, minimum height=1.5cm]
\tikzstyle{line} = [draw, arrows={-Triangle[length=0.2cm]}]

\tikzstyle{line_s} = [draw, ->]
\tikzstyle{line_st} = [draw,thick, ->]

\begin{tikzpicture}
  % draw the grid and the numbers
  \draw (-1,-1) grid (9,9) foreach \i in {0,...,9}{
    (\i-.5,9.5) node{\i} (-1.5,8.5-\i) node{\i}};

  \node[text width=1.5cm] at (2.0,7.5) (A_coord) {\Large \textbf{A}};
  \node[text width=1.5cm] at (5.0,7.5) (B_coord) {\Large \textbf{B}};
  \node[text width=1.5cm] at (5.0,4.5) (C_coord) {\Large \textbf{D}};
  \node[text width=1.5cm] at (2.0,4.5) (D_coord) {\Large \textbf{C}};

  \path [line, blue] (1.75, 7.5) node[transition, yshift=0.3cm, xshift=+1.2cm] {width} -- (4.25, 7.5);
  \path [line, blue] (1.50, 7.25) node[transition, yshift=-1.5cm, xshift=-0.6cm] {height} -- (1.5cm, 4.75);
}

\end{tikzpicture}
    }
\caption{Reprezentacija pravougaonika u obeležju.}
\label{rect_repr_graph}
\end{figure}

Sa slike(\ref{rect_repr_graph}) mogu se videti potrebni parametri za
reprezentaciju pravougaonika. \\
Kao referentu koordinatu izabrana je tačka A od koje se meri širina i visina
pravougaonika kao na slici. \\
Kako je za adresiranje podatka iz RAM-a potrebna linearna adresa kao što je
objašnjeno u sekciji(\ref{addr_trans_sec}) koordinata tačke A je linearizovana u
softveru. \\
Ostale koordinate moguće je dobiti na sledeći način:

\newpage

\setlength{\belowdisplayskip}{0pt} \setlength{\belowdisplayshortskip}{0pt}
\setlength{\abovedisplayskip}{0pt} \setlength{\abovedisplayshortskip}{0pt}

\begin{equation}
    \Scale[1.1]{ B = A * width}
    \label{rects_calc_coords_B}
\end{equation}
\begin{equation}
  \Scale[1.1]{ D = (A + width) + (height * FRAME\_WIDTH)}
  \label{rects_calc_coords_D}
\end{equation}
\begin{equation}
  \Scale[1.1]{ C = (A + width) + (height * FRAME\_WIDTH) - width}
  \label{rects_calc_coords_C}
\end{equation}


\newcolumntype{P}[1]{>{\centering\arraybackslash}p{#1}}
\begin{center}
  \centering
  \begin{tabular}{| P{1.5cm} | P{2.5cm} |}
    \hline
    N & Packed value(20 bit) \\ \hline
    0 &0x1A989 \\ \hline
    1 &0x1A987 \\ \hline
    2 &0x39249 \\ \hline
    3 &0x72926 \\ \hline
    $.$ & $.$ \\
    $.$ & $.$ \\
    $.$ & $.$ \\ \hline
    feature num - 1 &0x088d6 \\ \hline
  \end{tabular}
  \captionof{table}{Struktura zapakovane memorije rect\_ROM0 za model \ref{haarcascade_frontal_sec}} \label{rect_rom_packed_struct}
\end{center}

\newcolumntype{P}[1]{>{\centering\arraybackslash}p{#1}}
\begin{center}
  \centering
    \begin{tabular}{| P{1.5cm} | P{2cm} | P{2cm} | P{4cm} |}
    \hline
     N  & Height (5bit) & Width (5bit) & A linear coord (10bit) \\ \hline
      0 & 9 & 12 & 106 \\ \hline
      1 & 7 & 12 & 106 \\ \hline
      2 & 9 & 18 & 228 \\ \hline
      3 & 19 & 9 & 458 \\ \hline
      $.$ & $.$ & $.$ & $.$ \\
      $.$ & $.$ & $.$ & $.$ \\
      $.$ & $.$ & $.$ & $.$ \\ \hline
      feature num - 1 & 22 & 6 & 34 \\ \hline
    \end{tabular}
    \captionof{table}{Struktura raspakovane memorije rect\_ROM0 za model \ref{haarcascade_frontal_sec}} \label{rect_rom_unpacked_struct}
\end{center}

Na tabeli(\ref{rect_rom_packed_struct}) je prikazana strukura zapakovane memorije
rect\_ROM \\
Dok je tabeli(\ref{rect_rom_unpacked_struct}) prikazana polja raspakovane
memorije rect\_ROM \\
U prvoj koloni \textbf{N} su adrese lokacija, kojih ima features\_num (2913 u modelu
\ref{haarcascade_frontal_sec}). \\
Na prvoj tabeli drugoj koloni \textbf{Packed value(20 bit)} je zapakovana vrednost memorijske lokacije na
adresi u heksadecimalnom zapisu. \\
Na drugoj tabeli na drugoj i trećoj koloni se nalaze \textbf{height} i \textbf{width}
raspakovane vrednosti visine i širine pravougaonika. \\
Na drugoj tabeli i poslednjoj koloni se nalazi \textbf{A linear coord} raspakovana vrednost
linearne koordinate A. \\

Pored rect\_ROM memorije potrebna je weight\_ROM memorija koja će čuvati
vrednosti težina za svaki pravougaonik.

\begin{figure}[H]
  \centering
  \resizebox{0.9\textwidth}{!}{%
    \pgfdeclarelayer{background}
\pgfdeclarelayer{layer1}
\pgfdeclarelayer{layer2}
\pgfdeclarelayer{foreground}
\pgfsetlayers{background, layer2, layer1 ,main,foreground}

\tikzstyle{cloud1} = [draw=black, thick, fill=red!20, minimum height = 1em]
\tikzstyle{block_l} =[draw, text centered, fill=blue!15, minimum width=3.0cm, minimum height=2.5cm]
\tikzstyle{block_m} =[draw, text centered, fill=blue!15, minimum width=2.5cm, minimum height=1.5cm]
\tikzstyle{block_s} =[draw, text centered, fill=blue!15, minimum width=2cm, minimum height=1.5cm]
\tikzstyle{line} = [draw, arrows={-Triangle[length=0.2cm]}]

\begin{tikzpicture}[thick]
  \node [block_m] (feature_counter) {feature\_counter};
  \node [coordinate, right =0 cm of feature_counter] (feature_counter_out){};
  \node [coordinate, right =0.5 cm of feature_counter_out] (feature_counter_out_port){};

  \node [block_l, above right = 0.5cm and 2.0cm of feature_counter] (rect_rom) {rect\_ROM};
  \node [coordinate, left =0 cm of rect_rom] (rect_rom_in){};
  \node [coordinate, left =1 cm of rect_rom_in ] (rect_rom_in_port){};
  \node [coordinate, right =0 cm of rect_rom] (rect_rom_out){};
  \node[below] at (rect_rom.south) {\textbf{3x}};

  \node [block_l, below right = 0.5cm and 2.0cm of feature_counter] (weight_rom) {weight\_ROM};
  \node [coordinate, left =0 cm of weight_rom ] (weight_rom_in){};
  \node [coordinate, left =1 cm of weight_rom_in ] (weight_rom_in_port){};
  \node [coordinate, right =0 cm of weight_rom] (weight_rom_out){};
  \node [coordinate, right =8.5 cm of weight_rom_out] (weight_rom_out_port){};
  \node[above, yshift=0.4cm] at (weight_rom.north) {\textbf{3x}};

  \node [block_l, minimum height = 2cm, fill=red!15, right = 2.5cm of rect_rom] (calc_coords) {calc\_coords};
  \node [coordinate, left =0 cm of calc_coords ] (calc_coords_in){};
  \node [coordinate, right =0 cm of calc_coords ] (calc_coords_out){};
  \node [coordinate, right =3 cm of calc_coords_out ] (calc_coords_out_port){};
  \node[below] at (calc_coords.south) {\textbf{3x}};

  % arrows
  \path [line] (feature_counter_out) -- (feature_counter_out_port) -| (rect_rom_in_port) -- (rect_rom_in);
  \path [line] (feature_counter_out) -- (feature_counter_out_port) -| (weight_rom_in_port) -- (weight_rom_in);

  \path [line] (rect_rom_out) node[transition, yshift=0.3cm, xshift=+1.45cm] {\small packed\_data} -- (calc_coords_in);
  \path [line] (calc_coords_out) -- (calc_coords_out_port) node[transition, yshift=0.3cm, xshift=-1.2cm] {\textbf{rects\_addr}};
  \path [line] (weight_rom_out) -- (weight_rom_out_port) node[transition, yshift=0.3cm, xshift=-1.0cm] {\textbf{weights\_data}};

  % group

 \begin{pgfonlayer}{layer1}

   \node [block_l, below right = 0.5cm and 2.0cm of feature_counter, yshift=0.2cm, xshift=0.2cm] (weight_rom1) {weight\_ROM};
   \node [block_l, minimum height = 2cm, fill=red!15, right = 2.5cm of rect_rom, yshift=0.2cm, xshift=0.2cm] (calc_coords1) {calc\_coords};
   \node [block_l, above right = 0.5cm and 2.0cm of feature_counter, xshift=0.2cm ,yshift=0.2cm] (rect_rom1) {};
 \end{pgfonlayer}

 \begin{pgfonlayer}{layer2}

   \node [block_l, below right = 0.5cm and 2.0cm of feature_counter, yshift=0.4cm, xshift=0.4cm] (weight_rom2) {weight\_ROM};
   \node [block_l, minimum height = 2cm, fill=red!15, right = 2.5cm of rect_rom, yshift=0.4cm, xshift=0.4cm] (calc_coords2) {calc\_coords};
   \node [block_l, above right = 0.5cm and 2.0cm of feature_counter, xshift=0.4cm ,yshift=0.4cm] (rect_rom2) {};
 \end{pgfonlayer}


 \begin{pgfonlayer}{background}
  \node[inner sep=15pt, fill=yellow!25, rounded corners, inner ysep =30pt, draw, thick,fit=(feature_counter)(rect_rom) (weight_rom) (calc_coords) ] (frame_buffer) {};
  \node[above left] at (frame_buffer.south east) {\huge \textbf{features\_mem}};
  % \node[below right] at (frame_buffer.north west) {\huge \textbf{$3x$}};
\end{pgfonlayer}

\end{tikzpicture}
    }
\caption{Blok dijagram features\_mem modula.}
\label{features_mem_bd}
\end{figure}

Na slici(\ref{features_mem_bd}) je prikazan uprošćen blok dijagram features\_mem
modula. \\
Komponenta \textbf{feature\_counter} generiše adresu za čitanje iz ROM memorija.
\\
Postoje po 3 \textbf{rect\_ROM} i \textbf{weight\_ROM} memorije prethodno
opisane, po jedna za svaki pravougaonik u obeležju. \\
Komponenta \textbf{calc\_coords} vrši raspakivanje memorije opisane u
tablici(\ref{rect_rom_packed_struct}) na način opisan
formulama(\ref{rects_calc_coords_B}, \ref{rects_calc_coords_D},
\ref{rects_calc_coords_C}}). \\

Konačno može se izračunati potrebna memorija u slučaju modela opisanog u
sekciji(\ref{haarcascade_frontal_sec}). \\

\newcommand{\featureNum}{2913}
\newcommand{\rectNum}{3}
\newcommand{\wWeight}{3}
\newcommand{\wRect}{20}
\FPeval{\result}{clip(\featureNum*\rectNum*\wWeight*\wRect)}

\begin{center}
  \centering
    \begin{tabular}{| P{3.0cm} | P{3.0cm}|}
    \hline
     \textbf{feature\_num} & \featureNum{}  \\ \hline
     \textbf{rect\_num} & \rectNum{}  \\ \hline
     \textbf{w\_weight} & \wWeight{} bits  \\ \hline
     \textbf{w\_rect} & \wRect{} bits  \\ \hline
      \cellcolor{green!35} \textbf{TOTAL} & \cellcolor{green!35} \num[group-separator={,}]{\result{}} bits \\ \hline
    \end{tabular}
    \captionof{table}{Veličina memorije features\_mem za model \ref{haarcascade_frontal_sec}} \label{feature_mem_size}
\end{center}

\newpage

\subsection{Modul classifier}

\textbf{Classifier} modul obavlja samu klasifikaciju prozora.
Pored ii\_gen(\ref{ii_sii_gen_sec}) modula najviše utiče na performanse sistema. \\

Na slici ispod je prikazan blok dijagram klasifikatora.\\

\begin{figure}[H]
  \centering
  \resizebox{0.9\textwidth}{!}{%
    \pgfdeclarelayer{background1}
\pgfdeclarelayer{background2}
\pgfdeclarelayer{foreground}
\pgfsetlayers{background1, background2,main,foreground}

\tikzstyle{cloud1} = [draw=black, thick, fill=purple!20, minimum height = 1em]
\tikzstyle{block_l} =[draw, text centered, fill=blue!15, minimum width=2.5cm, minimum height=2.5cm]
\tikzstyle{block_m} =[draw, text centered, fill=blue!15, minimum width=2.0cm, text width=2.0cm, minimum height=2.0cm]
\tikzstyle{block_s} =[draw, text centered, fill=blue!15, minimum width=2cm, minimum height=1.5cm]
\tikzstyle{line} = [draw, arrows={-Triangle[length=0.2cm]}]

\tikzstyle{dreg} = [draw, text centered, fill=blue!15, minimum width = 1.5cm,
minimum height=1.5cm]
\tikzstyle{w_sum} = [draw, text centered, fill=blue!15, minimum width = 3.2cm,
minimum height=1cm]

\tikzstyle{square} = [draw, thick, fill=blue!15, circle, minimum size = 1cm, node distance = 2cm]
\tikzstyle{mult} = [draw, thick, fill=blue!15, circle, minimum size = 1cm ,node distance = 2cm]
\tikzstyle{sum} = [draw, thick, fill=blue!15, circle, minimum size= 1cm, node distance = 2cm]
\tikzstyle{operator} = [draw, thick, fill=blue!15, circle, minimum size= 1cm, node distance = 2cm]

\tikzset{
  multiplexer/.style={
    draw,
    trapezium,
    fill=blue!15,
    shape border uses incircle,
    shape border rotate=180,
    minimum size=25pt
  }
}

\begin{tikzpicture}[thick]

  \node [w_sum] (w_sum0) {weighted\_sum0};
  \node [coordinate, left=0cm of w_sum0] (w_sum0_in){};
  \node [coordinate, right=0cm of w_sum0] (w_sum0_out){};
  \node [coordinate, left=0.75cm of w_sum0_in] (w_sum0_in_port){};

  \node [w_sum, below=0.5cm of w_sum0] (w_sum1) {weighted\_sum1};
  \node [coordinate, left=0cm of w_sum1] (w_sum1_in){};
  \node [coordinate, right=0cm of w_sum1] (w_sum1_out){};
  \node [coordinate, left=0.75cm of w_sum1_in] (w_sum1_in_port){};
  \node [coordinate, left=3.0cm of w_sum1_in_port] (w_sum_in){};

  \node [w_sum, below=0.5cm of w_sum1] (w_sum2) {weighted\_sum2};
  \node [coordinate, left=0cm of w_sum2] (w_sum2_in){};
  \node [coordinate, right=0cm of w_sum2] (w_sum2_out){};
  \node [coordinate, left=0.75cm of w_sum2_in] (w_sum2_in_port){};

  \node [sum, right=0.75cm of w_sum1] (sum0) {+};
  \node [coordinate, above=0cm of sum0] (sum0_in0){};
  \node [coordinate, left=0cm of sum0] (sum0_in1){};
  \node [coordinate, below=0cm of sum0] (sum0_in2){};
  \node [coordinate, right=0cm of sum0] (sum0_out){};

  \node [operator, right=1.75cm of sum0] (leaf_lt) {\large \textbf{$<$}};
  \node [coordinate, left=0cm of leaf_lt] (leaf_lt_in0){};
  \node [coordinate, below=0cm of leaf_lt] (leaf_lt_in1){};
  \node [coordinate, right=0cm of leaf_lt] (leaf_lt_out){};

  \node [operator, below=2.5cm of leaf_lt] (mult0) {\large \textbf{$*$}};
  \node [coordinate, left=0cm of mult0] (mult0_in0){};
  \node [coordinate, below=0cm of mult0] (mult0_in1){};
  \node [coordinate, above=0cm of mult0] (mult0_out){};

  \node [coordinate, left=10.50cm of mult0_in0] (stddev){};

  \node [block_m, below right=2cm and -0.5cm of w_sum2] (feat_thr) {feature\_thr ROM};
  \node [coordinate, left=0cm of feat_thr] (feat_thr_addr){};
  \node [coordinate, right=0cm of feat_thr] (feat_thr_out){};

  \node [block_s, minimum width =2.5cm, left=1cm of feat_thr] (feature_cnt) {feature\_cnt};
  \node [coordinate, right=0cm of feature_cnt] (feature_addr){};

  \node [block_m, below=0.5cm of feat_thr] (stage_thr) {stage\_thr ROM};
  \node [coordinate, left=0cm of stage_thr] (stage_thr_addr){};
  \node [coordinate, right=0cm of stage_thr] (stage_thr_out){};

  \node [block_s, minimum width =2.5cm, left=1cm of stage_thr] (stage_cnt) {stage\_cnt};
  \node [coordinate, right=0cm of stage_cnt] (stage_addr){};

  \node [multiplexer, right=1.5cm of leaf_lt] (mux0) {};
  \node [coordinate, left=0cm of mux0] (mux0_sel){};
  \node [coordinate, above left=0cm and 0cm of mux0] (mux0_in0){};
  \node [coordinate, above right=0cm and 0cm of mux0] (mux0_in1){};
  \node [coordinate, below=0cm of mux0] (mux0_out){};


  \node [block_m, above left=1.5cm and 0cm of mux0] (leaf0) {leaf0\_ROM};
  \node [coordinate, above=0cm of leaf0] (leaf0_addr){};
  \node [coordinate, above=0.5cm of leaf0_addr] (leaf0_addr_port){};
  \node [coordinate, left=4cm of leaf0_addr_port] (leaf_feat_addr){};
  \node [coordinate, below=0cm of leaf0] (leaf0_out){};
  \node [coordinate, below=0.5cm of leaf0_out] (leaf0_port){};

  \node [block_m, above right=1.5cm and 0cm of mux0] (leaf1) {leaf1\_ROM};
  \node [coordinate, above=0cm of leaf1] (leaf1_addr){};
  \node [coordinate, above=0.5cm of leaf1_addr] (leaf1_addr_port){};
  \node [coordinate, below=0cm of leaf1] (leaf1_out){};
  \node [coordinate, below=0.5cm of leaf1_out] (leaf1_port){};

  \node [block_l, minimum height = 2cm, below=2.0cm of mux0] (accum_stage) {accum\_stage};
  \node [coordinate, below=0cm of accum_stage] (accum_stage_out){};
  \node [coordinate, above=0cm of accum_stage] (accum_stage_in){};

  \node [operator, below=2.585cm of accum_stage] (stage_lt) {\large \textbf{$<$}};
  \node [coordinate, above=0cm of stage_lt] (stage_lt_in0){};
  \node [coordinate, left=0cm of stage_lt] (stage_lt_in1){};
  \node [coordinate, below=0cm of stage_lt] (stage_lt_out){};
  \node [coordinate, below=4cm of stage_lt_out] (detect_out){};

  \node [block_s, fill=red!30, below left=1cm and 2.5cm of stage_lt] (local_rst) {local\_rst};
  \node [coordinate, right=0cm of local_rst] (local_rst_in){};

  % paths
  \path [line] (w_sum0_out) -| (sum0_in0);
  \path [line] (w_sum1_out) -- (sum0_in1);
  \path [line] (w_sum2_out) -| (sum0_in2);

  \path [line] (w_sum1_in_port) -- (w_sum0_in_port) -- (w_sum0_in);
  \path [line] (w_sum1_in_port) -- (w_sum1_in);
  \path [line] (w_sum1_in_port) -- (w_sum2_in_port) -- (w_sum2_in);
  \path [line, very thick] (w_sum_in) node[transition, yshift=+0.3cm, xshift=0.6cm] {\textbf{weight\_data}}  node[transition, yshift=-0.3cm, xshift=-0.0cm] {\textbf{fb\_ii}}  -- (w_sum1_in_port);

  \path [line] (feature_addr) node[transition, yshift=-1.0cm, xshift=-0.2cm] {feature\_addr}  -- (feat_thr_addr);
  \path [line] (feat_thr_out) -| (mult0_in1);
  \path [line] (stddev) node[transition, yshift=+0.3cm, xshift=0.6cm] {\textbf{stddev}} -- (mult0_in0);

  \path [line] (sum0_out) -- (leaf_lt_in0) node[transition, yshift=+0.8cm, xshift=-0.3cm] {rect\_sum};
  \path [line] (mult0_out) node[transition, yshift=+1.0cm, xshift=-0.8cm] {feat\_thr} -- (leaf_lt_in1);

  \path [line] (leaf_lt_out) -- (mux0_sel) node[transition, yshift=-0.5cm, xshift=-0.7cm] {leaf\_num};

  \path [line] (leaf0_out) -- (leaf0_port) -| (mux0_in0);
  \path [line] (leaf1_out) -- (leaf1_port) -| (mux0_in1);

  \path [line] (mux0_out) node[transition, yshift=-1.0cm, xshift=+0.8cm] {leaf\_val} -- (accum_stage_in);

  \path [line] (accum_stage_out) -- (stage_lt_in0);
  \path [line] (stage_addr) -- (stage_thr_addr);
  \path [line] (stage_thr_out) -- (stage_lt_in1);

  \path [line] (stage_lt_out) |- (local_rst_in);

  \path [line] (stage_lt_out) -- (detect_out) node[transition, yshift=+0.6cm, xshift=0.8cm] {\textbf{detect}} ;

  % \path [line] (leaf1_addr) |- (leaf0_addr_port) -- (leaf_feat_addr);
  \path [line] (leaf_feat_addr) node[transition, yshift=-0.3cm, xshift=1.2cm] {feature\_addr} -- (leaf1_addr_port) -- (leaf1_addr);
  \path [line] (leaf_feat_addr) -- (leaf0_addr_port) -- (leaf0_addr);

  %group

  \begin{pgfonlayer}{background2}
    \node[inner sep=15pt, fill=pink!25, rounded corners, draw, thick,fit=(w_sum0) (w_sum1) (w_sum2) (sum0) (w_sum1_in_port)] (weighted_sum) {};
    \node[below left] at (weighted_sum.north east) {weighted\_sum};
  \end{pgfonlayer}

  \begin{pgfonlayer}{background1}
    \node[inner sep=15pt, fill=yellow!25, rounded corners, draw, thick,fit=(feature_cnt) (feature_thr) (stage_cnt) (stage_thr) (local_rst) (weighted_sum) (accum_stage) (mux0) (leaf_lt) (mult0) (stage_lt) (leaf0) (leaf1) (leaf1_addr_port) (leaf0_addr_port) (leaf_feat_addr)] (frame_buffer) {};
    \node[below right] at (frame_buffer.north west) {\Huge \textbf{Classifier}};
  \end{pgfonlayer}

\end{tikzpicture}
    }
\caption{Blok dijagram classifier modula.}
\label{classifier_bd}
\end{figure}

\newpage

Sledeći primeri u C-u približno opisuje algoritam rada klasifikatora.

\begin{lstlisting}[language=C++,caption={Weighted\_sum u \textbf{C}-u},captionpos=b, label=weighted_sum_code]
  int weights[RECT_NUM][FEATURE_NUM]; //feature weights
  int rects[RECT_NUM][FEATURE_NUM][4];//unpacked rect_coords

  long weighted_sum_i(int ii[FRAME_HEIGHT][FRAME_WIDTH],
                      int f,      // feature_num
                      int r){     // rect_num

    long sum = ii[rects[r][f][0][1]][rects[r][f][0][0]] +
               ii[rects[r][f][3][1]][rects[r][f][3][0]] -
               ii[rects[r][f][2][1]][rects[r][f][2][0]] -
               ii[rects[r][f][1][1]][rects[r][f][1][0]];
    sum *= weights[r][f];

    return sum;
  }
  long weighted_sum(int ii[FRAME_HEIGHT][FRAME_WIDTH],
                    int feature_num){
    long sum0 = weighted_sum_i(ii, feature_num, 0);
    long sum1 = weighted_sum_i(ii, feature_num, 1);
    long sum2 = weighted_sum_i(ii, feature_num, 2);

    return sum0 + sum1 + sum2;
  }
\end{lstlisting}

Primer \ref{weighted_sum_code} približno prikazuje algoritam rada weighted\_sum
komponente sa slike(\ref{classifier_bd}).
Memorije weights i rects se nalaze u features\_mem komponenti u hardveskoj
implementaciji i objašnjene su u sekciji(\ref{features_meme_sec}).\\

Funkcija weighted\_sum\_i() računa sumu piksela pravougaonika na integralnoj slici.
Kao na slici(\ref{IntegralImage_img2}) na integralnoj slici potrebno je sabrati
gornji levi i donji desni piksel zatim oduzeti gornji desni i donji levi
piksel.\\
Pošto svaki pravougaonik ulazi u konačan zbir sa nekom težinom potrebno je
pomnožiti sumu pravougaonika sa težinom kao u liniji 12. \\

Funkcija weighted\_sum() dodatno sabira težinske sume sva tri pravougaonika.

\newpage

\begin{lstlisting}[language=C++,caption={Leaf\_val u \textbf{C}-u},captionpos=b, label=leaf_val_code]
  int feature_thresholds[FEATURE_NUM];
  int leaf_val0[FEATURE_NUM];
  int leaf_val1[FEATURE_NUM];

  int leaf_val(long stddev,
               long sum,
               int feature_num){

    if(sum <= feature_thresholds[feature_num] * stddev)
        return leaf_val0[feature_num];
    else
        return leaf_val1[feature_num];
  }
\end{lstlisting}

Primer(\ref{leaf_val_code}) prikazuje algoritam računanja leaf\_val vrednosti. \\
Memorija feature\_thresholds se nalazi na slici(\ref{classifier_bd}) pod nazivom
feature\_thr ROM. \\
Memorije leaf\_val0 i leaf\_val1 su leafVal0 i leafVal1 na
slici(\ref{classifier_bd}). \\

Funkcija leaf\_val kao ulaz prima standardnu devijaciju prozora, težinsku sumu
sva tri pravougaonika i broj trenutnog obeležja. \\
Povratna vrednost ove funkcije je sadržaj jedne od memorija leaf\_val0 ili
leaf\_val1 za to obeležje. \\

Ukoliko je težinska suma manja od praga obeležja feature\_threshold
pomnoženog sa standardnom devijacijom, povratna vrednost će biti iz memeorije
leaf\_val0, u suprotnom povratna vrednost će biti iz memorije leaf\_val1. \\

Dalje je potrebno akumulirati vrednosti unutar etape i uporediti sa pragom
etape.
Ukoliko je akumulirana vrednost manja od praga etape može se zaključiti da se na
posmatranom prozoru ne nalazi objekat.
Ukoliko je akumulirana vrednost veća od praga etape na posmatranom prozoru se
može naći objekat i potrebno je izvršiti sledeću etapu. \\

\begin{lstlisting}[language=C++,caption={Leaf\_val u \textbf{C}-u},captionpos=b, label=leaf_val_code]
  int stage_res(int leaf_val,
                int feature_start,
                int feature_end){
  int64_t sum = 0;

  for(int feature_num = feature_start;
      feature_num < feature_end;
      feature_num++){
    sum += leaf_val;
  }

  if(sum < stageThresholds[stage_num]) return -1;
  else return 1;
}
\end{lstlisting}

U hardverskoj implementaciji ukoliko je akumulirana vrednost etape manja od
praga etape, aktiviraće se local\_rst modul i resetovati \textbf{classifier}
modul, ovo će signalizirati ostatku sistema da se na posmatranom prozoru ne
nalazi objekat. \\

\subsection{Interfejsi}

Komunikacija svih unutrašnjih modula i komunikacija IP jezgra sa okolinom
realizovana je pomoću \textbf{\gls{dti}}\cite{PyGears_OSDA} interfejsa. \\
Korišćenjenjem ovog interfejsa dobijena je robusnija komunikacija između
modula zahvaljujući \emph{handshaking}-u. \\

Dodatna prednost ovog interfejsa je kompatibilnost sa \textbf{AXI-Stream}\cite{AXI_INTF}, \textbf{Avalon-ST}\cite{AVALON_INTF}
i sličnim streaming interfejsima. Zahvaljujući tome moguće je povezati ovaj
interfejs sa AXI-Stream i Avalon-ST bez dodatnog adaptera. \\

Interfejs će biti detaljnije objašnjen u PyGears(\ref{pygears_sec}) sekciji.