\section{Integracija IP jezgra sa Zynq sistemom}

PyGears i SystemVerilog implementacije predložene arhitekture su kompatibilne u
pogledu spoljnih interfejsa.
Pored toga ove implementacije su uveliko nezavisne od ciljane tehnologije za
implementaciju. \\
U ovom radu IP jezgro će biti implementirano na Zynq \gls{soc} sistemu. \\

Korišćeni DTI interfejs je kompatibilan sa AXI-Stream intefejsom, tako da je
integracija sa sistemima koji koriste druge interfejse poput Avalon-a trebala
biti jednostavna uz adaptere. \\

Preporučena je integracija IP jezgra u sistem sa procesorom, što će biti i
prikazano u ovom radu.
U ovom slučaju biće korišćen ARM Cortex-A9 procesor koji se može naći kao Hard
IP jezgro unutar Zynq SoC platforme kompanije Xilinx. \\

\subsection{Zynq}

Zynq se sastoji od \gls{ps} sistem i \gls{pl} sistema.
PS sistem je sačinjen od već pomenutog ARM Cortex-A9 \gls{APU} procesora.
Dok PL sistem predstavlja \gls{fpga} programabilnu logiku Artix-7 familije. \\

Komunikacija između PL i PS sistema se obavlja preko \gls{axi} interfejsa,
interapt pinova ili \gls{emio} pinova.\\

ARM procesor ima mogućnost povezivanja eksterne \gls{ddr} memorije preko
internog DDR kontrolera.
Preko ovog kontrolera pristup memoriji ima i PL sistem. \\
Zynq SoC ima podršku GNU/Linux operativnog sistema.

\subsection{Predloženi blok dijagram sistema}


\begin{figure}[H]
  \centering
  \resizebox{1\textwidth}{!}{%
    \pgfdeclarelayer{background1}
\pgfdeclarelayer{background2}
\pgfdeclarelayer{foreground}
\pgfsetlayers{background1, background2,main,foreground}

\tikzstyle{cloud1} = [draw=black, thick, fill=red!20, minimum height = 1em]
\tikzstyle{block_l} =[draw, text centered, fill=blue!15, minimum width=2.5cm, minimum height=2.5cm]
\tikzstyle{block_m} =[draw, text centered, fill=blue!15, minimum width=2.0cm, minimum height=1.0cm]
\tikzstyle{block_s} =[draw, text centered, fill=blue!15, minimum width=2cm, minimum height=1.5cm]
\tikzstyle{line} = [draw, line width = 0.05cm, arrows={-Triangle[length=0.2cm]}]

\tikzstyle{dreg} = [draw, text centered, fill=blue!15, minimum width = 1.5cm,
minimum height=1.5cm]

\tikzstyle{square} = [draw, thick, fill=blue!15, circle, minimum size = 1cm, node distance = 2cm]
\tikzstyle{mult} = [draw, thick, fill=blue!15, circle, minimum size = 1cm ,node distance = 2cm]
\tikzstyle{sum} = [draw, thick, fill=blue!15, circle, minimum size= 1cm, node distance = 2cm]

\begin{tikzpicture}[thick]

  \node [block_l, fill=red!20] (csc_cls) {\textbf{\large cascade\_classifier}};
  \node [coordinate, left = 0cm of csc_cls] (csc_cls_in){};
  \node [coordinate, left = 2cm of csc_cls_in] (ii_in){};
  \node [coordinate, above right = -0.5cm and 0cm of csc_cls] (csc_cls_out){};
  \node [coordinate, below right = -0.5cm and 0cm of csc_cls] (csc_cls_irq){};
  \node [coordinate, right = 1cm of csc_cls_irq] (csc_cls_irq1){};

  \node [block_l, fill=red!20, below = 2cm of csc_cls] (dm_cmd) {\textbf{\large dm\_cmd}};
  \node [coordinate, below left = 0 and -0.5cm of dm_cmd] (dm_cmd_in){};
  \node [coordinate, below right = 0 and -0.5cm of dm_cmd] (dm_cmd_irq_out){};
  \node [coordinate, above right = 0cm and -0.5cm of dm_cmd] (dm_cmd_irq_in){};
  \node [coordinate, above = 1cm of dm_cmd_irq_in] (dm_cmd_irq_in1){};
  \node [coordinate, left = 0cm of dm_cmd] (mm2s_cmd){};
  \node [coordinate, right = 0cm of dm_cmd] (s2mm_cmd){};

  \node [block_l, below = 2cm of dm_cmd] (apu) {\textbf{\large APU}};
  \node [coordinate, above left = 0cm and -0.5cm of apu] (apu_cmd){};
  \node [coordinate, above right = 0cm and -0.5cm of apu] (apu_irq){};
  \node [coordinate, left = 0cm of apu] (apu_mm2s){};
  \node [coordinate, right = 0cm of apu] (apu_s2mm){};

  \node [block_l, right = 3cm of csc_cls] (s2mm) {\textbf{\large dm\_s2mm}};
  \node [coordinate, right = 0cm of s2mm] (s2mm_out){};
  \node [coordinate, right = 1cm of s2mm] (s2mm_out1){};
  \node [coordinate, above left = -0.5cm and 0cm of s2mm] (s2mm_in){};
  \node [coordinate, below left = -0.5cm and 0cm of s2mm] (s2mm_cfg){};
  \node [coordinate, left = 1cm of s2mm_cfg] (s2mm_cfg1){};


  \node [block_l, left = 2cm of csc_cls] (mm2s) {\textbf{\large dm\_mm2s}};
  \node [coordinate, above left = -0.5cm and 0cm of mm2s] (mm2s_in){};
  \node [coordinate, left = 2cm of mm2s_in] (mm2s_in1){};
  \node [coordinate, below left = -0.5cm and 0cm of mm2s] (mm2s_cfg){};
  \node [coordinate, left = 1cm of mm2s_cfg] (mm2s_cfg1){};
  \node [coordinate, right = 0cm of mm2s] (mm2s_out){};

  %% group %%
 \begin{pgfonlayer}{background1}
   \node[inner sep=20pt, fill=yellow!25, rounded corners, draw, thick,fit=(csc_cls) (s2mm_out1) (mm2s_in1) (dm_cmd) (mm2s) (s2mm) (apu)] (system) {};
  \node[above left] at (system.south east) {\huge \textbf{SoC}};
\end{pgfonlayer}


  % arrows
  \path [line] (csc_cls_out) node[transition, yshift=+0.3cm, xshift=+1.3cm] {detect\_addr}  -- (s2mm_in);
  \path [line] (mm2s_out) -- (csc_cls_in) node[transition, yshift=+0.3cm, xshift=-0.8cm] {img\_in}  ;
  \path[line] (csc_cls_irq) node[transition, yshift=+0.3cm, xshift=+0.6cm] {irq}  -- (csc_cls_irq1) |- (dm_cmd_irq_in1) -- (dm_cmd_irq_in);

  \path [line] (s2mm_cmd) node[transition, yshift=+0.3cm, xshift=+1.0cm] {s2mm\_cfg} -| (s2mm_cfg1) -- (s2mm_cfg);
  \path [line] (mm2s_cmd) node[transition, yshift=+0.3cm, xshift=-1.0cm] {mm2s\_cfg} -| (mm2s_cfg1) -- (mm2s_cfg);

  \path[line] (s2mm_out) node[transition, yshift=+0.3cm, xshift=+0.8cm] {s2mm}  -- (s2mm_out1) |- (apu_s2mm);

  \path[line] (apu_mm2s) node[transition, yshift=+0.3cm, xshift=-0.8cm] {mm2s} -| (mm2s_in1) -- (mm2s_in);

  \path[line] (apu_cmd) node[transition, yshift=+0.5cm, xshift=-1.0cm] {apu\_cmd}  -- (dm_cmd_in);

  \path[line] (dm_cmd_irq_out) node[transition, yshift=-0.5cm, xshift=+0.4cm] {irq}  -- (apu_irq);


\end{tikzpicture}
  }
  \caption{Blok dijagram stddev modula.}
  \label{stddev_bd}
\end{figure}