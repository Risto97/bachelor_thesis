\section{Zaključak}

U ovom radu je uspešno projektovana arhitektura hardverskog akceleratora
Viola-Jones algoritma.
Performanse hardverskog rešenja nisu dostigle performanse OpenCV softverske
implementacije. \\

Predloženim optimizacijama moguće je dostići približne performanse OpenCV
implementacije.
Bolje performanse bi se postigle i korišćenjem većeg i kvalitetnijeg FPGA čipa,
na kojem bi se sistem implementirao sa višom frekvencijom takta.
U slučaju \gls{asic} implementacije moguće je dostići mnogo više
frekvencije rada, pa dobiti značajno brže rešenje od softverskog. \\

Tokom implementacije hardverske arhitekture korišćen je inovativni pristup opisa
hardvera korišćenjem PyGears metodologije.
Takođe ista arhitektura je implementirana i u standardnoj RTL SystemVerilog
implementaciji.
Zaključeno je da funkcionalni pristup koji nameće PyGears metodologija može
značajno ubrzati razvoj u nekim situacijama, dok je pojedine komponente
jednostavnije implementirati RTL metodologijom. \\

U radu je uspešno implementiran sistem na Zynq SoC platformi, prilikom
čega je bilo potrebno napisati Linux Device Driver.
Isto tako i konfigurisati i kompajlirati U-Boot i Linux Kernel za potrebe
projektovanog embeded sistema. \\
Napisana korisnička aplikacija u dva hijerarhijska nivoa, omogućava jednostavnu
komunikaciju sa projektovanim Kernel Driver-om iz različitih programskih jezika.
U ovom slučaju najviši nivo hijerarhije korisničke aplikacije je napisan u
Python-u. \\