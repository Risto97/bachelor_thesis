\section{Poređenje performansi}

Performanse projektovanog IP jezgra biće upoređene sa softverskim
implementacijama.
Porediće se C++ specifikacija napisana u ovom radu, OpenCV implementacija i
hardverska implementacija. \\
Mereno je samo vreme izvršavanja detekcije, odnosno isti zadatak koji obavlja IP
jezgro obalja i softver. \\
OpenCV implementacija biće poređena sa automatskim skaliranjem slike i sa
fiksnim faktorom skaliranja korišćenim u hardverskoj implementaciji. \\

Poređene platforme su Ryzen 5 1600 procesor pod Arch Linux-om, Thinkpad T430s
i5-3210m pod Arch Linux-om i Zynq 7020 pod Arch Linux ARM-om. \\

U sledećoj tabeli je prikazano prosečno vreme detekcije na Caltech dataset-u
slika veličine 240x320. \\

\newcolumntype{P}[1]{>{\centering\arraybackslash}p{#1}}
\begin{center}
  \centering
  \captionof{table}{Poređenje performansi}
  \begin{tabular}{| P{2cm} | P{2.5cm} | P{2.5cm} | P{2.5cm} | P{2.5cm} |}
    \hline
    Platform & OpenCV Scaling Auto & OpenCV Scaling Fixed & C++ Spec & IP Core \\ \hline
    Zynq & 571 ms & 205 ms & 3593 ms & 926 ms \\ \hline
    Thinkpad & 28 ms & 10.3 ms & 194 ms & N/A \\ \hline
    Ryzen & 22.8 ms & 9.7 ms & 192 ms & N/A \\ \hline
  \end{tabular}
\end{center}

Kao što je bilo i očekivano OpenCV implementacija pod Ryzen procesorom ima
najkraće vreme izvršavanja.
Može se videti da iako Ryzen procesor ima 6 bržih CPU jezgara za razliku od 2 kod
Thinkpad laptopa ne postoje drastične razlike u vremenu izvršavanja softverskih
implementacija. \\
Zynq je očekivano najsporiji od tri platforme.
Projektovano IP jezgro je 4.5 puta sporije od OpenCV implementacije na istoj
platformi. \\

Iako su performanse IP jezgra slabije od OpenCV implementacije treba uzeti u
obzir i odnos frekvencija procesora i IP jezgra.
Zynq 7020 se sastoji od dva Cortex-A9 procesora taktovana sa 866MHz, što znači 8.5
puta veća frekvencija takta od IP jezgra.
Pored toga projektovano IP jezgro zauzima samo oko 15\% hardverskih resursa Zynq
čipa, što znači da postoji prostora za ubrzanjem paralelizmom. \\

Prednost FPGA implementacije leži i u tome što je procesor značajno
manje opterećen prilikom detekcije, pa je procesor slobodan da radi ostale
zadatke u paraleli.
Pošto IP jezgra zauzimaju malo hardverskih resursa, postoji mogućnost
instancioniranja više IP jezgra u sistemu i paralelne detekcije više slika
istovremeno, pritom bez gubitaka performansi usled paralelizma. \\