\section{Sažetak}

Cilj ovog rada je hardverska implementacija akceleratora Viola-Jones algoritma. \\
Pored toga hardverska implementacija je urađena pomoću dve metodologije kako bi
se utvrdile prednosti i mane ovih metodologija. \\

\noindent
Kako bi se projektovao željeni hardverski akcelerator bilo je potrebno odraditi
sledeće stvari: \\
\begin{itemize}
  \item Pisanje implementacija u Python i C++ programskim jezicima kao
    specifikacija za izvršavanje, koja pritom pomaže prilikom particionisanja i projektovanja hardvera.
  \item Projektovanje arhitekture hardverskog akceleratora za Viola-Jones algoritam.
  \item Pisanje HDL modela za sintezu u SystemVerilog RTL metodologiji i
    \PyGears{} metodologiji.
  \item Poređenje dve metodologije i analiza prednosti i mana obe metodologije.
  \item Implementacija projektovanog IP jezgra na \ZTurn{} ploči sa Zynq
    7020 SoC.
  \item Pisanje Linux Kernel drajvera i korisničke aplikacije za korišćenje
    jezgra za detekciju lica na Xilinx Zynq platformi.

\end{itemize}