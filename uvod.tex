\section{Motivacija}

Automatizacijom industrijskih poslova i razvojem digitalnih sistema, detekcija objekata na
slici postaje sve zastupljenija u mnogobrojnim uređajima.
Uređaji koji imaju potrebe za detekcijom objekata na slici se mogu naći u
industrijskoj elektronici, medicinskoj, potrošačkoj, automobilskoj industriji
itd... \\

Detekcija objekata je danas sve češće prisutna u embeded sistemima.
Kako u ovim sistemima postoji ograničenje resursa, potrošnje i performansi
potrebno je pronaći algoritam koji zadovoljava zadata ograničenja.
Dobitak performansi u ovim sistemima često dolazi sa cenom potrebnih resursa za
implementaciju a samim tim i višom cenom. \\

Viola-Jones algoritam predstavlja jedan od algoritama za detekciju objekata koji
se može efikasno implementirati na embeded sistemima.
Algoritam implementira tehnike koje smanjuju memorijski ``futprint'' i ubrzavaju
računanje uz minimalan gubitak pouzdanosti.
Zbog toga je ovaj algoritam izuzetno pogodan za hardversku implementaciju. \\

\noindent
U ovom radu biće prikazano:
\begin{itemize}
\item Teorijski uvod u Viola-Jones algoritam.
\item Arhitektura digitalnog hardverskog akceleratora algoritma.
\item Integracija akceleratora u Zynq 7020 SoC sistem.
\item Vremenska analiza implementacije i ubrzavanje akceleratora skraćivanjem kombinacionih
  putanja.
\item Poređenje akceleratora sa softverskim implementacijama algoritma.
\item Analiza potrošnje resursa PyGears i SystemVerilog implementacija.
\item Objašnjenje projektovanog Linux Kernel Driver-a i korisničkih aplikacija.
\end{itemize}